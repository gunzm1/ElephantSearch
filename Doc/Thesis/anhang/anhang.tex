\begin{titlepage}


    \clearpage
    \vspace*{\fill}
    \begin{center}
        \begin{minipage}{.6\textwidth}
            \fontsize{26pt}{28pt}\selectfont
            Anhang
        \end{minipage}
    \end{center}
    \vfill % equivalent to \vspace{\fill}
    \clearpage


\end{titlepage}

\newpage 

% In den Anhang fügen Sie ein:
%  * Details des Projektpans, falls vorhanden
%  * Resultate und Zwischenresultate in Funktion der Projektiterationen
%  * Pflichtenheft / Anforderungsspezifikation (Stand Ende dritter Woche)
%  * Angaben zum Projektrepository
%  * Sitzungsprotokolle, falls vorhanden
%  * Weiterführende Erläuterungen zu den verwendeten Technologien, falls nötig
%  * Benutzerhandbuch, falls vorhanden und sinnvoll, es hier aufzulisten
%  * Installations- und Betriebsdokument, falls vorhanden und sinnvoll, es hier aufzulisten
% Unterlassen Sie das Anfügen von Listings.

\appendix

\section*{Anforderungen}
\label{sec:anhang:anforderungen}
\href{anhang/anforderungen.pdf}{Anforderungsdokument}

\section*{Beispiele}
\label{sec:anhang:sparql_beispiele}
\href{anhang/schnipsel.pdf}{Dokument mit diversen Abfragebeispielen}

\section*{Dokumentation Wissensmodellierung}
\label{sec:anhang:tutorial_dokument}
\href{../Tutorial/template.pdf}{Dokumentation Wissensmodellierung}

\section*{Dokumentation BTI7302 --- Projekt2}
\label{sec:anhang:projekt2}
\href{../Extern/EigeneDokumente/DokumentationProjekt2.pdf}{Dokumentation BTI7302 - Projekt2}

\section*{Arbeitsjournal}
\label{sec:anhang:journal}
TODO:\@ Journal hier einfügen.

\section*{Benuterhandbuch}
\label{sec:anhang:handbuch}
TODO:\@ Handbuch hier einfügen.
