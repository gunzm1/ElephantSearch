\chapter{Fazit}
\label{chap:fazit}

% In diesem Kapitel legen Sie Ihre kritische Betrachtung Ihrer Arbeit dar. Zeigen Sie, was Sie erreicht haben. Hinterfragen Sie das Ergebnis auf objektive Art und Weise.

Im Kapitel Lösung wurde bereits festgestellt, dass sämtliche Ziele der Arbeit erreicht wurde. Dabei ist wichtig, dass wir im Laufe der Arbeit die Problemdomäne ändern geändert haben. Die Ursprünglich gewählte Domäne des Erlernen der Programmiersprache Prolog stellte sich als nicht geeignet heraus. Als wir dieses Gebiet auswählten wussten wir noch zu wenig über die Wissensmodellierung mittels Ontologien bescheid. So war uns nicht bewusst, dass sich diese vorallem für solche Gebiete eignet in denen viele Fakten vorhanden sind. Das Erlernen der Programmiersprache Prolog ist ein sehr theoretisches gebiet, mit Abläufen und verallgemeinerungen. Im Gegensatz dazu eignet sich das Plannen von Reisen hervorragend. Damit war es uns möglich die verschiedenen Spezialitäten eines Expertensystems auszutesten und die Mächtigkeit an Semantischen Datenbanken auf zu zeigen.

Bei Unserer Arbeit lag der Fokus nicht auf einem genau definierten Endprodukt. Viel mehr stand der gesamte Prozess der Wissenserarbeitung im Vordergrund. Ein wichtiger Teil war die Modellierung, vor allem wie sinnvoll vorgegangen werden kann um eine Ontologie aufzubauen. Das ist auch der Grund weshalb die verschiedenen Schwierigkeiten, auf welche wir gestossen sind als Teil des Prozesses, also als Erfahrung und nicht als Nachteil angesehen werden muss. 

Wir haben uns bewusst für ein für uns eher unbekanntes Thema der Thesis entschieden. Dies vorallem, weil wir in der erarbeitung der Bachelorthesis eine Chance sahen zu Forschen und zu experimentieren. In der Berufswelt, welche jetzt auf uns wartet wird dies wahrscheinlich nur noch begrenzt möglich sein. Uns ist aber erst beim Reflektieren der Arbeit richtiggehend bewusst geworden, dass wir ein Themengebiet gewählt haben bei dem wir ausser einigen simplen Grundlagen alles neu Erarbeiten mussten. Es war uns zu keinem Zeitpunkt möglich, vorhandenes Wissen anzuwenden. Dies führte zu einer sehr spannenden und intensiven Arbeit.

%Zusammenfassung des Resultats
%Würdigung des Resultats
%Evtl. Empfehlungen zum weiteren Vorgehen

%- Alle Ziele erreicht\\
%- aber Problemdomäne musste geändert werden\\
%- Ontologien machen nur sinn wo genügend Fakten vorhanden sind.
%- grosser Fokus auf der Modellierung; wie mach ich das am besten. Am anfang gewisse schwierigkeiten; haben wir aber in der griff bekommen\\
%- Wissensmodellierung hat vor und nachteile:\\
%VOrteile: html schnittstelle; mit vorhandenen Werkzeugen wird es übersichtlicher (Protege)\\
%Nachteile: Regeln mit Indiviuen nicht möglicht; Rechnen nur eingeschränkt möglich\\
%- Struktur der ARbeit sinnvoll; mit erarbeiten der Grundlagen und Praktischer Umsetztung
%
%
%- mögliche Weiterführung: Ontologie erweitern; dynamische distanzten (Googlemaps); Zeiteinheit verbessern; Könnte so wirklich genutzt werden
%
%- Persönliches Feedback: Erfahrungen und Schwierigkeiten
%
%Vorteile und Nachteile
%+ Sehr klare Trennung Eigenschaften, Klassen und Relationen => klares Raster
%+ Verwendung von Logik / Ableitung von Regeln
%+ Mischform UML und Prolog
%- Abläufe sind sehr schwer abzubilden
%- 
%
%
%
%
%Ein grosser Vorteil im Vergleich zu anderen Expertensystemen ist sicher, dass die OWL / XML Schreibweise gut verarbeitet werden kann und es sich so anbietet HTML-Schnittstellen zu genieren und zu nutzen.  (TODO soll hier gesagt werden was?). 
