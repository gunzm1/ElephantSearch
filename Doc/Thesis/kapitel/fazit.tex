\chapter{Fazit und Ausblick}
\label{chap:fazit}

% In diesem Kapitel legen Sie Ihre kritische Betrachtung Ihrer Arbeit dar. Zeigen Sie, was Sie erreicht haben. Hinterfragen Sie das Ergebnis auf objektive Art und Weise.

Im Kapitel Lösung wurde bereits festgestellt, dass sämtliche Ziele der Arbeit erreicht wurden. Dabei ist wichtig, dass wir im Laufe der Arbeit die Problemdomäne ändern geändert haben. Die ursprünglich gewählte Domäne des Erlernens des Programmierens anhand der Programmiersprache Prolog stellte sich als nicht geeignet heraus.

Als wir dieses Gebiet auswählten, wussten wir noch zu wenig über die Wissensmodellierung mittels Ontologien. So war uns nicht bewusst, dass sich diese vor allem für solche Gebiete eignet, in denen viele Fakten vorhanden sind. Das Erlernen des Programmierens anhand Prolog ist ein sehr theoretisches Gebiet mit Abläufen und Verallgemeinerungen. Im Gegensatz ist die das Planen von Reisen viel besser dazu geeignet. Damit war es uns möglich die verschiedenen Spezialitäten eines Expertensystems herauszufinden und die Mächtigkeit von semantischen Datenbanken aufzuzeigen.

Bei unserer Arbeit lag der Fokus nicht auf einem genau definierten Endprodukt. Viel mehr stand der gesamte Prozess der Wissenserarbeitung im Vordergrund. Ein wichtiger Teil war die Modellierung einer Ontologie. Hierbei war es schwierig ein sinnvolles, allgemeines Vorgehen zu finden und definieren. So sehen wir die verschiedenen Schwierigkeiten, auf welche wir während des Prozesses gestossen sind, nicht als Nachteil sondern als Erfahrung an.

Wir haben uns bewusst für ein für uns eher unbekanntes Thema der Thesis entschieden. Dies vor allem, weil wir in der Erarbeitung der Bachelor Thesis eine Chance sahen zu forschen und zu experimentieren. In der Berufswelt, welche jetzt auf uns wartet wird dies wahrscheinlich nur noch begrenzt möglich sein. Uns ist aber erst beim Reflektieren der Arbeit richtiggehend bewusst geworden, dass wir ein Themengebiet gewählt haben bei dem wir ausser einigen simplen Grundlagen alles neu erarbeiten mussten. Es war uns nur sehr bedingt möglich, vorhandenes Wissen anzuwenden. Dies führte zu einer sehr spannenden und intensiven Arbeit.

\section{Ausblick}
\label{sec:fazit_subchap}
Bei dem aktuellen Modell der Ontologie sowie der Benutzeroberfläche handelt es sich um einen Nachweis der Machbarkeit in Form von Prototypen. Der Funktionsumfang dieser beschränkt sich daher nur auf das Nötigste und entspricht nicht einer realen Anwendung.

Wollte man das Umgesetzte einer praxisrelevanten Anwendung nutzen, so müsste dies erweitert werden. Die Ontologie verfügt nur über einige wenige, exemplarische Entitäten. Um interessante Abfragemöglichkeiten bieten zu können, wäre sicherlich eine starke Erweiterung der Ontologie nötig.

Weiter könnte die Anwendung mit Benutzerprofilen versehen werden, welche den Standort des Benutzers speichern. Mittels einer Geoinformationssoftware könnte so die Distanz zu Entitäten (z.B. Restaurants und Ausflüge) dynamisch berechnet werden.

Das Modell unterstützt aktuell nur eine Zeitauflösung von einem halben Tag, einem ganzen Tag sowie mehr als einem Tag. Eine feinere Zeitauflösung, etwa in Form von Fliesskommawerten, liesse eine genauere Planung von Reisen zu. Dies müsste jedoch in Form von Applikationslogik und nicht durch den Reasoner umgesetzt werden, da dieser, wie im Tutorial erwähnt, keine Bedingungserfüllungsprobleme lösen kann.

%- Wissensmodellierung hat vor und nachteile:\\
%   Vorteile: html schnittstelle; mit vorhandenen Werkzeugen wird es übersichtlicher (Protege)\\
%   Nachteile: Regeln mit Indiviuen nicht möglicht; Rechnen nur eingeschränkt möglich\\
%- Struktur der ARbeit sinnvoll; mit erarbeiten der Grundlagen und Praktischer Umsetztung
%Vorteile und Nachteile
%+ Sehr klare Trennung Eigenschaften, Klassen und Relationen => klares Raster
%+ Verwendung von Logik / Ableitung von Regeln
%+ Mischform UML und Prolog
%- Abläufe sind sehr schwer abzubilden
%Ein grosser Vorteil im Vergleich zu anderen Expertensystemen ist sicher, dass die OWL / XML Schreibweise gut verarbeitet werden kann und es sich so anbietet HTML-Schnittstellen zu genieren und zu nutzen.  (TODO soll hier gesagt werden was?). 




% - Zusammenfassung des Resultats
% - Würdigung des Resultats
% - Evtl. Empfehlungen zum weiteren Vorgehen
