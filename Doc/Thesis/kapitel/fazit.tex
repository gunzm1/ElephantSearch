\chapter{Fazit}
\label{chap:fazit}

% In diesem Kapitel legen Sie Ihre kritische Betrachtung Ihrer Arbeit dar. Zeigen Sie, was Sie erreicht haben. Hinterfragen Sie das Ergebnis auf objektive Art und Weise.

Im Kapitel Lösung wurde bereits festgestellt, dass sämtliche Ziele der Arbeit erreicht wurde. Dabei ist wichtig, dass wir im Laufe der Arbeit die Problemdomäne ändern 

%Zusammenfassung des Resultats
%Würdigung des Resultats
%Evtl. Empfehlungen zum weiteren Vorgehen

- Alle Ziele erreicht\\
- aber Problemdomäne musste geändert werden\\
- Ontologien machen nur sinn wo genügend Fakten vorhanden sind.
- grosser Fokus auf der Modellierung; wie mach ich das am besten. Am anfang gewisse schwierigkeiten; haben wir aber in der griff bekommen\\
- Wissensmodellierung hat vor und nachteile:\\
VOrteile: html schnittstelle; mit vorhandenen Werkzeugen wird es übersichtlicher (Protege)\\
Nachteile: Regeln mit Indiviuen nicht möglicht; Rechnen nur eingeschränkt möglich\\
- Struktur der ARbeit sinnvoll; mit erarbeiten der Grundlagen und Praktischer Umsetztung


- mögliche Weiterführung: Ontologie erweitern; dynamische distanzten (Googlemaps); Zeiteinheit verbessern; Könnte so wirklich genutzt werden

- Persönliches Feedback: Erfahrungen und Schwierigkeiten

