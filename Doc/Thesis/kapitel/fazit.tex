\chapter{Fazit und Erweiterungsmöglickeiten}
\label{chap:fazit}

% In diesem Kapitel legen Sie Ihre kritische Betrachtung Ihrer Arbeit dar. Zeigen Sie, was Sie erreicht haben. Hinterfragen Sie das Ergebnis auf objektive Art und Weise.

Die Ziele der Arbeit konnten erreicht werden, wie im~\autoref{chap:loesung} dargestellt. Zu erwähnen ist, dass wir im Verlaufe der Arbeit die Problemdomäne ändern mussten. Das Erlernen des Programmierens anhand der Programmiersprache Prolog stellte sich als ungeeignet heraus.

Bei der Auswahl hatten wir noch geringe Kenntnisse über die Wissensmodellierung mittels Ontologien. Uns war nicht bewusst, dass sich diese Form der Wissensmodellierung vor allem für solche Gebiete eignet, die aufgrund der Ontologie Inferenz und damit Schlussfolgerungen zulassen. Die ursprünglich gewählte Problemdomäne, das Erlernen des Programmierens anhand Prolog lag auf einer zu hohen Abstraktionsebene. Im Gegensatz dazu ist die schliesslich gewählte Problemdomäne, das Planen von Reisen viel besser geeignet. Damit war es uns möglich die verschiedenen Spezialitäten eines Expertensystems herauszufinden und die Mächtigkeit von semantischen Datenbanken aufzuzeigen.

Durch diese Arbeit erhielten wir einen Überblick über das Thema semantische Datenbanken. Dabei konnten wir eine mögliche Vorgehensweise bei der Modellierung von Wissen mittels Ontologien analysieren und umsetzen.

Mit Hilfe der eingesetzten Werkzeuge konnte die Modellierung im gewünschten Sinne umgesetzt werden. Hilfreich war beispielsweise die Strukturierung der Daten durch Protégé.
Überraschenderweise fanden wir in den beiden Werkzeugen Protégé und Stardog jedoch Fehler (was uns zunächst erhebliche Schwierigkeiten bereitete). Durch die Kombination der beiden Werkzeuge, konnten die Fehler jedoch umgangen werden.

Semantische Netze bieten eine zur Entwicklung freundliche Möglichkeit Wissen zu modellieren. Einerseits bieten sie einen Überblick während der Entwicklung, andererseits ist die Abbildung des Wissens in solchen Netzen auch für Laien gut verständlich.

Nutzt eine Applikation eine semantische Datenbank als Datenmodell, erfordern Anpassungen (Modellierungen) des Datenmodells keine Programmänderungen --- bei geschickter Programmierung. Modellierungen sind z.B.\ das Hinzufügen, Bearbeiten oder Löschen von Entitäten (Klassen, Individuen, Relationen oder Eigenschaften). Im Gegensatz hierzu benötigen Änderungen in relationalen Datenbanken meistens sehr aufwendige Programmänderungen.\\
Eine eher ungünstige Eigenart dieser Wissensabbildung ist, dass vollständige Prozesse und Abläufe nur schwer abzubilden sind. Dies kann durch andere Sprachen einfacher geschehen.

Bei unserer Arbeit lag der Fokus nicht auf einem definierten Endprodukt. Im Vordergrund stand viel mehr der gesamte Prozess der Wissenserarbeitung. Ein wichtiger Teil der Arbeit bestand in der Modellierung einer Ontologie. Hierbei war es schwierig ein sinnvolles und allgemein gültiges Vorgehen zu finden und zu definieren.\\
Verschiedenen Schwierigkeiten, auf welche wir während des Prozesses der Wissenserarbeitung gestossen sind, erachten wir als Bereicherung unserer Erfahrung.\\
Im Rahmen des Prozesses entstand der Prototyp einer Webapplikation zur benutzerfreundlichen Reiseplanung. Diese ermöglicht das Planen einer Reise mit Hilfe eines Assistenten. Der Benutzer kann die gewünschten Reisekriterien auswählen. Er erhält eine Auswahl an Reiseobjekten, welche seinen Kriterien entsprechen.\\
Weiter entstand ein Tutorial, welches die theoretischen Grundlagen der Wissensmodellierung erklärt und den Leser anhand eines Beispieles durch den gesamten Prozess der Wissensmodellierung führt.

Für die Bachelor-Thesis fiel unsere Entscheidung bewusst auf ein für uns unbekanntes Thema. Wir hofften dadurch, in der Erarbeitung der Bachelor-Thesis in einem in der Geschäftswelt nicht alltäglichen Gebiet forschen und experimentieren zu können. Mit Eintritt in die Berufswelt wird dies wahrscheinlich nur noch begrenzt möglich sein.\\
Erst beim Reflektieren der Arbeit wurde uns richtig bewusst, ein Themengebiet gewählt zu haben, bei dem wir ausser einigen einfachen Grundlagen alles neu erarbeiten mussten. Dies führte zu einer sehr spannenden und intensiven Arbeit.

% - Zusammenfassung des Resultats
% - Würdigung des Resultats
% - Evtl. Empfehlungen zum weiteren Vorgehen

\section{Erweiterungsmöglickeiten}
\label{sec:fazit_subchap}
Bei dem aktuellen Modell der Ontologie sowie der Benutzeroberfläche handelt es sich um einen Nachweis der Machbarkeit in Form von Prototypen. Der Funktionsumfang beschränkt sich daher nur auf das Nötigste und entspricht nicht einer realen Anwendung.

Für eine praxisrelevante Anwendung müsste das Umgesetzte erweitert werden. Um interessante Abfragemöglichkeiten anbieten zu können, müsste die umgesetzte Ontologie stark erweitert werden.

Die Anwendung könnte beispielsweise mit Benutzerprofilen versehen werden, welche Standort des Benutzers speichern und mittels Geoinformationssoftware die Distanz zu den Entitäten (Restaurants und Ausflüge) berechnen.

Bei der Zeitauflösung beschränkt sich das Modell auf halbtägige, ganztägige sowie mehrtägige Ausflüge. Eine feinere Zeitauflösung, zum Beispiel in Form von Gleitkommawerten, würde eine genauere Planung ermöglichen. Da ein Reasoner keine Bedingungserfüllbarkeitsproblem lösen kann, müsste dies in der Applikation umgesetzt werden.
