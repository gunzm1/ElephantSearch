\chapter{Lösungsansatz}
\label{chap:loesungsansatz}

% Hier beschreiben Sie Ihren Lösungsansatz. Der Lösungsansatz ist ein Beschrieb auf hoher Abstraktionsebene. Beschreiben Sie, falls nötig, die zum Einsatz gekommenen Technologien nur dann, wenn die Beschreibung für das Verständnis der Arbeit unbedingt notwendig ist.
In der heutigen Zeit werden Ferien häufig per Internet gebucht. Was aber, wenn der Urlaub nicht einfach zwei Wochen an einem Ort stattfinden soll? Was, wenn der Kunde reisen möchte? Oder sonstige spezielle Wünsche hat? Für solche Anforderungen muss er auch heute noch ins Reisebüro um sich beraten zu lassen.

Um auch diesen Prozess zu automatisieren soll eine Ontologie erstellt werden, welche mithilfe von Eigenschaften, Kriterien und Regeln verschiedene Reisevorschläge machen kann.
Vorgehen
Damit dieses Ziel erreicht werden kann, muss zuerst eine Ontologie in Form von Klassen, Individuen, Relationen und Eigenschaften erstellt werden. Dies kann jedoch schnell ins Uferlose übergehen, wenn kein klarer Rahmen definiert ist. Daher wird anhand von exemplarischen Reisen die Ontologie schrittweise aufgebaut.

Konrekt werden Beispiele von Reisen mit diversen Anforderungen genannt, welche dann Stück für Stück modelliert werden, so dass schlussendlich eine vollständige Ontologie entsteht. Ein Beispiel solch einer Reise kann z.B. eine vierwöchige Abenteuerreise für Singles quer durch den Amazonas, verbunden mit einem abschliessenden Aufenthalt in einem Wellness-Ressort. Dabei darf das Budget beispielsweise eine gewisse Limite nicht überschreiten.
