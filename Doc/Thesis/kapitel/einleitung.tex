\chapter{Einleitung}
\label{chap:einleitung}

Bei dieser Arbeit handelt es sich um den Aufbau einer semantischen Datenbank sowie der praktischen Nutzung dieser. Die Arbeit wurde in zwei Teilen umgesetzt: Einem eher theoretischen und einem eher praktischen Teil. Der eher theoretische Teil zeigt in Form eines Dokumentes auf, wie ein Knowledge Engineer bei der Wissensmodellierung in Form von Ontologien mit Hilfe von semantischen Datenbanken vorgeht. Bei dem eher praktischen Teil handelt es sich um eine praktische Umsetzung des theoretischen Teils in Form eines Reiseplaners durch eine minimale Webanwendung.

Die gestecken Ziele konnten allesamt erreicht werden.

% Hier beschreiben Sie kurz das Thema der Arbeit, den Kontext sowie in knappen Worten das Ergebnis. Dieser Abschnitt gleicht einem Management Summary. Ziel dieses Kapitels ist, dem Leser die Entscheidungsgrundlage zu liefern, ob er die Arbeit lesen soll oder nicht.

% Einträge im Verzeichnis erscheinen lassen ohne hier eine Referenz einzufügen
%\nocite{kopka:band1}

In der heutigen Zeit werden Ferien häufig per Internet gebucht. Was aber, wenn der Urlaub nicht einfach zwei Wochen an einem Ort stattfinden soll? Was, wenn der Kunde reisen möchte? Oder sonstige spezielle Wünsche hat? Für solche Anforderungen muss er auch heute noch in ein Reisebüro um sich beraten zu lassen.

Um auch diesen Prozess zu automatisieren soll eine Ontologie erstellt werden, welche mithilfe von Eigenschaften, Kriterien und Regeln verschiedene Reisevorschläge machen kann.

Damit dieses Ziel erreicht werden kann, muss zuerst eine Ontologie in Form von Klassen, Individuen, Relationen und Eigenschaften erstellt werden. Dies kann jedoch schnell ins Uferlose übergehen, wenn kein klarer Rahmen definiert ist. Daher wird anhand von exemplarischen Reisen die Ontologie schrittweise aufgebaut.

Konkret werden Beispiele von Reisen mit diversen Anforderungen genannt, welche dann Stück für Stück modelliert werden, so dass schlussendlich eine vollständige Ontologie entsteht. Ein Beispiel solch einer Reise kann z.B. eine vierwöchige Abenteuerreise für Singles quer durch den Amazonas, verbunden mit einem abschliessenden Aufenthalt in einem Wellness-Ressort. Dabei darf das Budget beispielsweise eine gewisse Limite nicht überschreiten.

