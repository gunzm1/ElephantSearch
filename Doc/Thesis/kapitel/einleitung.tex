\chapter{Einleitung}
\label{chap:einleitung}

%CHeckliste fausibibel
%Ausgangslage, Problemstellung
%• Auftrag oder Aufgabenstellung
%• Resultat der Arbeit
%• Methodik (Herkunft der Daten, Vorgehen, Systemund
%Gültigkeitsgrenzen)
%• Folgerungen, offene Fragen

Der klassische Ansatz der Wissensabbildung, zum Beispiel in Form von UML, welchem die relationale Datenspeicherung zugrunde liegt, wird in der heutigen Informatik weitläufig eingesetzt und ist de facto Standard. Häufig geschieht dies in enger Verbindung mit der objektorientierten Programmierung. Experten aus einer Fachrichtung sind fähig diese Daten zu interpretieren und daraus Schlüsse zu ziehen. Es ist aber nicht möglich automatisch Fragestellungen zu beantworten, welche über reine Relationsverknüpfungen hinausgehen. Mit dieser Technik sind Objekteigenschaften und -Verhalten also eher schwer abbildbar. Eine andere Art Wissen zu repräsentieren sind semantische Datenbanken. Diese ermöglichen das Abbilden des Objektverhaltens und können mithilfe von Schlussfolgerungen die Rolle des Experten einnehmen.

In dieser Bachelorthesis soll eine solche semantische Datenbank aufgebaut und angewendet werden.  Die Arbeit wurde in zwei Teilen umgesetzt: Einem  theoretischen und einem praktischen Teil. Der theoretische Teil zeigt in Form eines Tutorials auf, wie ein knowledge engineer bei der Wissensmodellierung vorgeht. Er nutzt dabei Ontologien als Basis, um eine semantische Datenbank aufzubauen. Im praktischen Teil soll eine solche Ontologie aufgebaut und per Benutzerschnittstelle zugänglich gemacht werden.

Die gesteckten Ziele konnten allesamt erreicht werden. Es ist den Autoren gelungen die Theorie der Wissensmodellierung übersichtlich in einem Dokument aufzubereiten und wiederzugeben. Speziell dabei ist, dass in diesem fachliche Grundlagen mit einem praktischen Beispiel verbunden werden. Zusätzlich konnten auch konkrete Tipps aus der eigenen Erfahrung eingeflochten werden. Dieses Dokument ist dem Abschlussdokument als Anhang beigefügt.

Im praktischen Teil der Arbeit wurde ein Expertensystem für Reisen aufgebaut: ``In der heutigen Zeit werden Ferien häufig per Internet gebucht. Was aber, wenn der Urlaub nicht einfach zwei Wochen an einem Ort stattfinden soll? Was, wenn der Kunde reisen möchte? Oder sonstige spezielle Wünsche hat? Für solche Anforderungen muss er auch heute noch in ein Reisebüro um sich beraten zu lassen.''\\
Bei der Automatisierung dieses Prozesses kommt die Ontologie ins Spiel, welche mithilfe von Eigenschaften, Kriterien und Regeln die Problemdomäne abbildet. Durch die Verwendung eines Reasoners können verschiedene Reisevorschläge gemacht werden.\\
Um eine sehr individuelle Reiseplanung zu erreichen, musste zuerst eine Ontologie in Form von Klassen, Individuen, Relationen und Eigenschaften erstellt werden. Ohne klaren Rahmen würde eine Ontologie zu komplex und zu gross. Daher bauten die Autoren die Ontologie anhand exemplarischer Tages und Wochenendausflügen in der Schweiz auf.\\
Als Ergebnis können die Autoren eine semantischen Datenbank präsentieren. Dabei wird die von ihnen gewählte Problemdomäne in einem klar gesteckten Rahmen abgedeckt und die Mächtigkeit der Wissensmodellierung veranschaulicht. Die Schnittstelle für Benutzer wurde in Form eines Assistenten umgesetzt, welcher den Benutzer durch die Möglichkeiten führt und ihn bei der Planung seiner Reise unterstützt. Damit konnten die Autoren verhindern, dass Benutzer gezwungen sind, Abfragen in einer komplexen Abfragesprache zu formulieren.

Die Wissensmodellierung auf Basis von Ontolgien ist auf ihre Weise eine mächtiges Werkzeug um Informationen mit Semantik zu versehen. Trotzdem hat diese Art der Modellierung gewisse Einschränkungen, welche von anderen auf Logik basierenden Sprachen, wie zum Beispiel Prolog, abgedeckt werden. Im Laufe der Arbeit erkannten die Autoren, dass es noch Bedarf an Weiterentwicklung im Bezug auf die angebotenen Werkzeuge gibt. So musste eine Kombination von zwei Werkzeugen verwendet werden um sämtliche Anforderungen umzusetzen.

% Hier beschreiben Sie kurz das Thema der Arbeit, den Kontext sowie in knappen Worten das Ergebnis. Dieser Abschnitt gleicht einem Management Summary. Ziel dieses Kapitels ist, dem Leser die Entscheidungsgrundlage zu liefern, ob er die Arbeit lesen soll oder nicht.

% Einträge im Verzeichnis erscheinen lassen ohne hier eine Referenz einzufügen
%\nocite{kopka:band1}
