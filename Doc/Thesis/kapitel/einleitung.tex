\chapter{Einleitung}
\label{chap:einleitung}

%CHeckliste fausibibel
%Ausgangslage, Problemstellung
%• Auftrag oder Aufgabenstellung
%• Resultat der Arbeit
%• Methodik (Herkunft der Daten, Vorgehen, Systemund
%Gültigkeitsgrenzen)
%• Folgerungen, offene Fragen

Der klassische Ansatz der Wissensabbildung, zum Beispiel in Form von UML, welchem die relationale Datenspeicherung zugrunde liegt, wird in der heutigen Informatik weitläufig eingesetzt und ist de facto Standard. Häufig geschieht dies in enger Verbindung mit der objektorientierten Programmierung. Experten aus einer Fachrichtung sind fähig diese Daten zu interpretieren und daraus Schlüsse zu ziehen. Es ist aber nicht möglich automatisch Fragestellungen zu beantworten, welche über reine Relationsverknüpfungen hinausgehen. Mit dieser Technik sind Objekteigenschaften und -verhalten also eher schwer abbildbar. Eine andere Art Wissen zu repräsentieren sind semantische Datenbanken. Diese ermöglichen das Abbilden dieser Eigenschaften und können mithilfe von Schlussfolgerungen die Rolle des Experten einnehmen. 
In dieser Bachelorthesis soll eine solche semantische Datenbank aufgebaut und angewendet werden.  Die Arbeit wurde in zwei Teilen umgesetzt: Einem eher theoretischen und einem eher praktischen Teil. Der eher theoretische Teil zeigt in Form eines Tutorials auf, wie ein Knowledge Engineer bei der Wissensmodellierung in Form von Ontologien mit Hilfe von semantischen Datenbanken vorgeht. Im praktischen Teil soll eine solche Ontologie aufgebaut und mittels eines GUIs für den Benutzer zugänglich machen.

Die gesteckten Ziele konnten allesamt erreicht werden. Es ist den Autoren gelungen, auf eine übersichtliche Art und Weise, die Theorie zu der Wissensmodellierung festzuhalten. Speziell dabei ist, dass in dem Dokument fachliche Grundlagen mit einem praktischen Beispiel verbunden werden. Zusätzlich konnten auch konkrete Tipps aus der eigenen Erfahrung eingeflochten werden. Dieses Dokument ist dem Abschlussdokument als Anhang beigefügt. 

Im praktischen Teil der Arbeit wurde ein Expertensystem für Reiseplaner aufgebaut: "`In der heutigen Zeit werden Ferien häufig per Internet gebucht. Was aber, wenn der Urlaub nicht einfach zwei Wochen an einem Ort stattfinden soll? Was, wenn der Kunde reisen möchte? Oder sonstige spezielle Wünsche hat? Für solche Anforderungen muss er auch heute noch in ein Reisebüro um sich beraten zu lassen."'\\
Bei der Automatisierung dieses Prozesses kommt die Ontologie ins Spiel, welche mithilfe von Eigenschaften, Kriterien und Regeln die Problemdomäne abbildet. Durch die Verwendung eines Reasoners können verschiedene Reisevorschläge gemacht werden.\\
Damit dieses Ziel erreicht werden konnte, musste zuerst eine Ontologie in Form von Klassen, Individuen, Relationen und Eigenschaften erstellt werden. Da diese schnell ins Uferlose übergehen kann, wurde ein klarer Rahmen definiert. An exemplarischen Tages und Wochenendausflügen in der Schweiz, konnten die Autoren die Ontologie schrittweise Aufbauen. \\
Als Ergebnis können die Autoren eine Semantischen Datenbank präsentieren, welche die von ihnen gewählte Problemdomäne in einem klar gesteckten Rahmen abdeckt und die Mächtigkeit der Wissensmodellierung veranschaulicht. Die Schnittstelle für Benutzer wurde in Form eines Assistenten umgesetzt, welcher den Benutzer durch die Möglichkeiten führt und in bei der Planung seiner Reise unterstützt. Des weiteren könnten die Autoren mit diesem Wizard verhindern, dass der Benutzer gezwungen ist, Abfragen in einer schlecht lesbaren Abfragesprache zu formulieren. 

Die Wissensmodellierung ist auf ihre Weise eine mächtiges Werkzeug um Daten mit Logik zu versehen. Ein grosser Vorteil im Vergleich zu anderen Expertensystemen ist sicher, dass die OWL / XML Schreibweise gut verarbeitet werden kann und es sich so anbietet html Schnittstellen zu genieren und zu nutzen. Trotzdem bietet diese Art der Modellierung gewisse Einschränkungen welche von anderen Logiksprachen wie Prolog abgedeckt werden (TODO soll hier gesagt werden was?). Im Laufe der Arbeit, erkannten die Autoren, dass es noch Bedarf an Weiterentwicklung im Bezug auf die angebotenen Werkzeuge gibt. So musste eine Kombination von zwei Werkzeugen verwendet werden um sämtliche Anforderungen umzusetzen.



% Hier beschreiben Sie kurz das Thema der Arbeit, den Kontext sowie in knappen Worten das Ergebnis. Dieser Abschnitt gleicht einem Management Summary. Ziel dieses Kapitels ist, dem Leser die Entscheidungsgrundlage zu liefern, ob er die Arbeit lesen soll oder nicht.

% Einträge im Verzeichnis erscheinen lassen ohne hier eine Referenz einzufügen
%\nocite{kopka:band1}
