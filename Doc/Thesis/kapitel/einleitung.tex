\chapter{Einleitung}
\label{chap:einleitung}

%CHeckliste fausibibel
%Ausgangslage, Problemstellung
%• Auftrag oder Aufgabenstellung
%• Resultat der Arbeit
%• Methodik (Herkunft der Daten, Vorgehen, Systemund
%Gültigkeitsgrenzen)
%• Folgerungen, offene Fragen

In der heutigen Informatik ist die relationale Datenspeicherung de facto Standard. Die Wissensabbildung erfolgt beispielsweise durch UML.\@ Häufig geschieht dies in enger Verbindung mit der objektorientierten Programmierung. Experten aus einer Fachrichtung sind fähig, die Daten zu interpretieren und Schlüsse zu ziehen. Es ist damit aber nicht möglich Fragestellungen zu beantworten, welche über reine Relationsverknüpfungen hinausgehen. Mit dieser Technik sind Objekteigenschaften und -verhalten schwer abbildbar. Semantische Datenbanken sind eine andere Möglichkeit Wissen zu repräsentieren. Sie ermöglichen das Abbilden von Objekteigenschaften und können mit Hilfe von Schlussfolgerungen die Rolle eines Experten einnehmen.

In dieser Bachelor-Thesis soll eine semantische Datenbank aufgebaut und angewendet werden.  Die Arbeit einthält zwei Teile: Einen theoretischen und einen praktischen. Der theoretische Teil (das Tutorial) zeigt auf, wie ein Knowledge-Engineer bei der Wissensmodellierung vorgehen könnte.\\
Um eine semantische Datenbank aufzubauen nutzt er als Basis Ontologien.\\
Im praktischen Teil wird eine entsprechende Ontologie aufgebaut und durch eine Benutzerschnittstelle zugänglich gemacht.

Im \textbf{theoretischen} Teil gelang es den Autoren, die Theorie der Wissensmodellierung übersichtlich aufzubereiten und wiederzugeben. Als Besonderheit werden fachliche Grundlagen mit praktischen Beispielen verbunden. Aus gemachten Erfahrungen konnten konkrete Hinweise eingeflochten werden. Dieses Dokument ist als Anhang beigefügt.

Im \textbf{praktischen Teil} der Arbeit wurde ein Expertensystem für Reisen entwickelt: ``In der heutigen Zeit werden Urlaubsaufenthalte/Pauschalreisen häufiger per Internet gebucht. Wenn jedoch ein Kunde spezielle Wünsche hat, so muss er sich immer noch in einem Reisebüro beraten lassen.''\\
Für die Automatisierung dieses Prozesses werden Ontologien verwendet. Diese bilden mit Hilfe von Eigenschaften, Kriterien und Regeln die Problemdomäne ab. Mit Hilfe eines Reasoners können unterschiedliche Reiseanfragen beantwortet werden. Für eine individuelle Reiseplanung, musste eine Ontologie erstellt werden.\\
Ohne klaren Rahmen würde diese zu komplex und zu gross. Daher haben die Autoren die Ontologie auf exemplarische Tages- und Wochenendausflüge in der Schweiz eingeschränkt.

Als \textbf{Ergebnis} präsentieren die Autoren eine semantischen Datenbank. In dieser wird die gewählte Problemdomäne in einem klar gesteckten Rahmen abgedeckt und die Vorteile der Wissensmodellierung werden veranschaulicht. Die Benutzerschnittstelle wurde durch einen Assistenten umgesetzt, der den Benutzer durch die (Reise-) Möglichkeiten führt ohne Kenntnisse der Ontologie und einer speziellen Abfragesprache zu erfordern.

% Hier beschreiben Sie kurz das Thema der Arbeit, den Kontext sowie in knappen Worten das Ergebnis. Dieser Abschnitt gleicht einem Management Summary. Ziel dieses Kapitels ist, dem Leser die Entscheidungsgrundlage zu liefern, ob er die Arbeit lesen soll oder nicht.

% Einträge im Verzeichnis erscheinen lassen ohne hier eine Referenz einzufügen
%\nocite{kopka:band1}
