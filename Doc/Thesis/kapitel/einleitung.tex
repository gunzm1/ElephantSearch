\chapter{Einleitung}
\label{chap:einleitung}

Das nachfolgende Dokument beschreibt die Anforderungen der Bachelorthesis von Sven Osterwalder und Mira Günzburger. Als Vorarbeit der Bachelorthesis dient die Arbeit über die Semantische Suche, welche im Rahmen des Moduls 7302 "`Projekt 2"' bereits erstellt wurde.

Wie in der Abschlussdokumentation der Projekt 2 Arbeit beschrieben, handelt es sich bei semantischen  Suchmaschinen um "`Werkzeuge, die in der Lage sind, auf Fragen mit Hilfe einer Datenbank oder des Internets Antworten zu generieren. Solche Werkzeuge können insbesondere dann eine sehr wertvolle Unterstützung für den menschlichen Experten sein, wenn unter extremer Zeitnot komplexe Entscheidungen getroffen werden müssen, wie beispielsweise in der medizinischen Diagnostik. Die Firma IBM hat vor nicht allzu langer Zeit für eine Überraschung gesorgt, als sie die Leistungsfähigkeit von „Watson“ im Quiz Jeopardy demonstriert hat. In diesem Quiz, wo schwierige, oft zweideutig formulierte Fragen aus beliebigen Bereichen unter Zeitdruck beantwortet werden müssen, konnte sich Watson überlegen gegenüber zwei bisher sehr erfolgreichen menschlichen Champions durchsetzen.
\cite{projekt2Doc}

Wie im Fazit der Projektarbeit beschrieben, muss der Fokus der Thesis verschoben werden. So soll im Rahmen der Bachelorthesis ein Dokument mit Tutorial-Charakter erstellt werden, welches einen leicht verständlichen Einstieg in das Thema der Semantischen Datenbanken/ Suchen bietet. Als Hilfsmittel dafür kommt Apache Stanbol zum Einsatz.

Nachfolgend werden die einzelnen Elemente der Bachelorthesis beschrieben.

\section{Aufgabenstellung}
\label{sec:Aufgabenstellung}
"`Ziel dieser Arbeit ist die Entwicklung und Anwendung eines Systems zur Speicherung in einer Semantischen Datenbank auf der Basis von Apache Stanbol. Dies schliesst die Erstellung einer Domänen-Ontologie mittels RDF/OWL ein und die Anwendung dieser Ontologie auf ein Anwendungsproblem, wie beispielsweise der Erlernung einer Programmiersprache ein. Exemplarisch soll aufgezeigt werden, wie dabei ein Knowledge Engineer vorgehen, um eine Problemdomäne systematisch zu modellieren und formalisieren. Besondere Bedeutung kommt dabei der Schnittstelle zwischen Mensch und System zu."'~\cite{Aufgabenstellung}

% Einträge im Verzeichnis erscheinen lassen ohne hier eine Referenz einzufügen
%\nocite{kopka:band1}
