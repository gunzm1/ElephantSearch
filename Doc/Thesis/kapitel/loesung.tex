\chapter{Lösung}
\label{chap:loesung}

% Hier beschreiben Sie Ihren Lösungsansatz. Der Lösungsansatz ist ein Beschrieb auf hoher Abstraktionsebene. Beschreiben Sie, falls nötig, die zum Einsatz gekommenen Technologien nur dann, wenn die Beschreibung für das Verständnis der Arbeit unbedingt notwendig ist.

 - Beanwortung von Fragen innerhalb einer gewissen Wissensdomäne

 - dazu müssen wir eine Ontolgie abbilden -> dies führt zu abhängikeiten wleche in implementation beschrieben werden.

% In diesem Kapitel beschreiben Sie Ihre Lösung des Problems. Geben Sie dem Leser genügend Einblick in die Lösung, so dass er Ihre Arbeit entsprechend würdigen kann. Verwenden Sie aber Anhänge für Dinge, die hier nicht unbedingt bis ins letzte Detail verstanden werden müssen.

Aufgabenstellung und Milestones analysieren und auswerten:

 - modelleriung der Ontologie
 - generieren von Abfragen (sqarql)
 - Benutzeroberfläche
 - exemplarisch aufzeigen wie ein Knowledge Engineer vorgeht (Tutorial)
\section{Modellierung}
\label{sec:modellierung}
Ontologie beschreiben: aber wie ausführlich? 
TODO: Sollen wir die Fehler die wir gefunden haben erwähnen; ich würde schon, aber wo? -> Hier ist gut


\section{Benutzeroberfläche}
\label{sec:gui}


