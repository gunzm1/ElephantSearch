\chapter{Administratives}
\label{chap:administratives}

Im folgende Kapitel sollen einige administrative Aspekte der Diplomarbeit angesprochen werden. Diese sind aber für das Verständnis der in der Arbeit erzeugten Ergebnisse nicht notwendig wurden aber der Vollständigkeit halber hinzugefügt.

\section{Sinn und Zweck}
\label{sec:admin_sinn}
TODO: Brauchen wir das vielleicht auch?

\section{Beteiligte Personen}
\label{sec:admin_beteiligte}
\begin{tabbing} %tabulator
Bezeichnung: \= Name name name name\= Zuständigkeit \kill \\
  Autoren           \>  Mira Günzburger   \> \\
										\> Sven Osterwalder    \> \\
	Betreuer           \> Dr. Jürgen Eckerle    \> Begleitet Studenten bei der Thesis\\
	Experte           \>  Jean-Marie Leclerc  \> Zuständig für die Beurteilung der Arbeit
\end{tabbing}

\section{Aufbau, Struktur und Sprache}
\label{sec:admin_aufbau}
Das Dokument beginnt mit einer Allgemeinen Einleitung zu der Bachelor Thesis. Nach der Beschreibung der Aufgabenstellung führen die Autoren vom Vorgehen, welche die Herangehensweise der Autoren bei der Arbeit beschreibt, zur finalen Lösung. Im Kapitel Technologien werden die verwendeten Technologien kurz vorgestellt. Da der Schwerpunkt dieser Thesis aber nicht beim Endprodukt sondern im Prozess der Entstehung und dem Sammeln von Erkenntnisse rund um das Thema liegt, steht der Fokus des Abschlussdokuments auch bei der tatsächlichen Umsetzung. 
Zusätzliche Dokumente, wie das zu einem späteren Zeitpunkt beschriebenen Tutorialdokument oder die konkrete Modellierung sind diesem Dokument als Anhang hinzugefügt.

Im gesamten Dokument wird aus Grund der literarischen Vereinfachung und der besseren Lesbarkeit nur die männliche Form verwendet. Mit dem Geschriebene sollen aber sowohl männliche als auch weibliche Personen angesprochen werden.

\section{Ergebnisse (Deliverables)}
\label{sec:admin_ergebniss}
Die Bachelor Thesis umfasst mehrere Dokumente, wobei es sich bei diesem um das Abschlussdokument der Arbeit handelt. Zur Übersicht werden in diesem Kapitel sämtliche abzugebende Objekte aufgeführt:
\begin{itemize}
	\item Abschlussdokument\\ Das Abschlussdokument führt sämtliche Ergebnisse zusammen und bietet eine Übersicht über den gesamten Arbeitsprozess
	\item Tutorial\\ Das Tutorial beinhaltet den theoretischen Aspekt der Arbeit
	\item Modellierung \\in Form von einem XML\textbackslash RDF Dokument\\ Grafische Abbildung des Modells
	\item Benutzeroberfläche\\ Source Code der Benutzeroberfläche mit Dokumentation
	\item Benutzerhandbuch 
\end{itemize}

\section{Zeitplan}
\label{sec:admin_zeitplan}
TODO: das können wir ja dann später mal fertig machen, das muss ja weder korrigert noch abgesegnet werden

\begin{center}
    \begin{tabular}{ | p{5cm} | l | l |l|l|l|l|l|l|l|l|l|l|l|l|l|l|} 
		\hline %  l| l | l | l | l | l | l | l | l | l | l | l | l | l |
    Anforderungsdokument 	& \cellcolor{cyan} &  \cellcolor{cyan}&  \cellcolor{cyan}& & & & & & & & & & & & & \\  \hline  % \cellcolor{green} % & & & & & & & & & & & & & & & 
    Tutorial  & & & & & & & & & & & & & & & & \\  \hline 
    Modell  & & & & & & & & & & & & & & & & \\ \hline
    Benutzeroberfläche  & & & & & & & & & & & & & & & & \\  \hline
    Abschlussdokument  & & & & & & & & & & & & & & & & \\   \hline
    Präsentation vorbereiten  & & & & & & & & & & & & & & & &  \\  \hline
    Verteidigung vorbereiten  & & & & & & & & & & & & & & & &  \\  \hline		
    \end{tabular}
\end{center}