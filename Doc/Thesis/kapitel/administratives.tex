\chapter{Administratives}
\label{chap:administratives}

Im folgende Kapitel sollen einige administrative Aspekte der Thesis angesprochen werden. Diese sind aber für das Verständnis der erzeugten Ergebnisse nicht notwendig, wurden aber der Vollständigkeit halber hinzugefügt.

\section{Beteiligte Personen}
\label{sec:admin_beteiligte}
\begin{tabbing} %tabulator
Bezeichnung: \= Name name name name\= Zuständigkeit \kill \\
    Autoren         \> Mira Günzburger\protect\footnotemark[1]{}    \> \\
                    \> Sven Osterwalder\protect\footnotemark[2]{} \> \\
    Betreuer        \> Prof. Dr.\ Jürgen Eckerle\protect\footnotemark[3]{}  \> \textit{Begleitet Studenten bei der Thesis}\\
    Experte         \> Jean-Marie Leclerc   \> \textit{Zuständig für die Beurteilung der Arbeit}
\end{tabbing}
\footnotetext[1]{mira.guenzburger@students.bfh.ch}
\footnotetext[2]{sven.osterwalder@students.bfh.ch}
\footnotetext[3]{juergen.eckerle@bfh.ch}

\section{Aufbau, Struktur und Sprache}
\label{sec:admin_aufbau}
Das Dokument beginnt mit einer Einleitung. Nach Beschreibung der Aufgabenstellung führen die Autoren vom Vorgehen, welches die Herangehensweise der Autoren bei der Arbeit beschreibt, zur finalen Lösung. Im Kapitel Komponenten werden die verwendeten Technologien vorgestellt. Der Schwerpunkt dieser Thesis liegt nicht im Endprodukt,  sondern im Prozess der Erstellung eines Expertensystems und dem Sammeln von Erkenntnisse rund um das Thema. Daher liegt der Fokus dieses Dokumentes auch bei der tatsächlichen Umsetzung. Zusätzliche Dokumente sind diesem Dokument als Anhang hinzugefügt. Es sind dies zum Beispiel das zu einem späteren Zeitpunkt beschriebene Tutorialdokument oder die konkrete Modellierung.

Im gesamten Dokument wird aus Grund der literarischen Vereinfachung und der besseren Lesbarkeit nur die männliche Form verwendet. Mit dem Geschriebenen sind aber sowohl männliche als auch weibliche Personen angesprochen.

\section{Ergebnisse (Deliverables)}
\label{sec:admin_ergebniss}
Die Bachelor-Thesis umfasst mehrere Dokumente. Bei dem hier vorliegenden handelt es sich um das Abschlussdokument der Arbeit. Der Übersicht halber werden in diesem Kapitel sämtliche abzugebende Objekte aufgeführt:
\begin{itemize}
	\item \textbf{Abschlussdokument} \\
        Das Abschlussdokument führt sämtliche Ergebnisse zusammen und bietet eine Übersicht über den gesamten Arbeitsprozess
	\item \textbf{Tutorial} \\
        Das Tutorial beinhaltet das praktische Vorgehen zur Umsetzung einer Modellierung sowie die theoretischen Grundlagen
	\item \textbf{Modellierung} \\
        in Form von einem XML/RDF Dokument, grafische Abbildung des Modells
	\item \textbf{Benutzeroberfläche} \\
        Quellcode der Benutzeroberfläche und Dokumentation
	\item \textbf{Benutzerhandbuch} \\
        Beschreibt Installation und Verwendung der semantischen Datenbank

\end{itemize}
