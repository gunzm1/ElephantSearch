\chapter{Administratives}
\label{chap:administratives}
Einige administrative Aspekte der Bachelor-Thesis werden angesprochen, obwohl sie für das Verständnis der Resultate nicht notwendig sind.

Im gesamten Dokument wird nur die männliche Form verwendet, womit aber beide Geschlechter gemeint sind.

\section{Beteiligte Personen}
\label{sec:admin_beteiligte}
\begin{tabbing} %tabulator
Bezeichnung: \= Name name name name name name\= Zuständigkeit \kill \\
    Autoren         \> Mira Günzburger\protect\footnotemark[1]{}    \> \\
                    \> Sven Osterwalder\protect\footnotemark[2]{} \> \\
    Betreuer        \> Prof.\ Dr.\ Jürgen Eckerle\protect\footnotemark[3]{}  \> \textit{Begleitet Studenten bei der Bachelor-Thesis}\\
    Experte         \> Jean-Marie Leclerc   \> \textit{Zuständig für die Beurteilung der Arbeit}
\end{tabbing}
\footnotetext[1]{mira.guenzburger@students.bfh.ch}
\footnotetext[2]{sven.osterwalder@students.bfh.ch}
\footnotetext[3]{juergen.eckerle@bfh.ch}

\section{Aufbau der Dokumentation}
\label{sec:admin_aufbau}
Der Aufbau der vorliegenden Arbeit ist wie folgt:
\begin{itemize}
    \item Einleitung zur Bachelor-Thesis
    \item Beschreibung der Aufgabenstellung
    \item Vorgehen der Autoren im Hinblick auf die gestellten Aufgaben
    \item Lösung der gestellten Aufgaben
    \item Verwendete Technologien
\end{itemize}

Die Schwerpunkte dieser Arbeit liegen in:
\begin{itemize}
    \item der Beschreibung der theoretischen Grundlagen (unter praktischen Aspekten) für ein Expertensystem
    \item der Entwicklung und Anwendung eines Expertensystems
\end{itemize}

\section{Ergebnisse (Deliverables)}
\label{sec:admin_ergebniss}
Die Bachelor-Thesis besteht aus mehreren Dokumenten. Vorgelegt wird das Abschlussdokument der Arbeit.

Nachfolgend sind die abzugebenden Objekte aufgeführt:
\begin{itemize}
	\item \textbf{Abschlussdokument} \\
        Das Abschlussdokument vereinigt sämtliche Ergebnisse und bietet eine Übersicht des gesamten Arbeitsprozesses
	\item \textbf{Tutorial} \\
        Das Tutorial beinhaltet die theoretischen Grundlagen (unter praktischen Aspekten) für ein Expertensystem
	\item \textbf{Modellierung des Expertensystems} \\
        in Form eines RDF/XML-Dokumentes und in Form einer grafischen Abbildung
	\item \textbf{Benutzeroberfläche} \\
        Quellcode und Dokumentation der umgesetzten Benutzeroberfläche
	\item \textbf{Installationshandbuch} \\
        Beschreibt Installation und Verwendung des entwickelten Expertensystems einschliesslich der Benutzeroberfläche
\end{itemize}
