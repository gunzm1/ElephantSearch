\chapter{Aufgabenstellung}
\label{chap:Aufgabenstellung}

% In diesem Kapitel wird die Aufgabenstellung beschrieben. Erwähnt werden soll die Ausgangslage (was hat es schon gegeben, was ist schon gemacht worden), auf der die Arbeit aufbaut. Im weiteren wird das Ziel beschrieben, welches erreicht werden soll. Daraus leitet sich das in dieser Arbeit gelöste Problem ab.

\section{Motivation}
\label{sec:aufgabe_motivation}
Um Wissen aus dem Internet zu beziehen verwendet man hauptsächlich Suchmaschinen. Sie werden von vielen Personen bei der täglichen Arbeit verwendet.

Die meisten Suchmaschinen erfordern die Eingabe von Begrifflichkeiten im Suchfeld. Sie indexieren Inhalte mittels Stich- und Schlagworten. Der Suchende muss daher zumindest eine bestimmte Vorstellung von den erwarteten Suchergebnissen haben.

Fragt man nach Konzepten und Zusammenhängen, so stösst man bei den heutigen Suchmaschinen schnell an deren Grenzen. Sie verfügen über geringes bis kein semantisches Wissen, sie kennen keine Konzepte. Um dennoch die gewünschten Ergebnisse zu erhalten, muss der Anwender entsprechende Stichworte als Eingabe zu einer Suche liefern.

Zwar existieren Mechanismen zur Abbildung von Konzepten und Zusammenhängen. Ihre Nutzung ist aber komplex.

In der vorliegenden Arbeit wurde versucht eine bestimmte Suchmaschine zu entwickeln, welche Wissen über Konzepte und über Zusammenhänge hat. Sie soll eine intuitive, einfach zu bedienende Schnittstelle zwischen Mensch und System bieten.

\section{Ausgangslage}
\label{sec:aufgabe_ausgangslage}
Grundlagen für die Arbeit waren den Autoren teilweise bekannt. An der Berner Fachhochschule für angewandte Wissenschaften existiert kein Hauptfach für semantische Datenbanken bzw. Semantik.

Für den Erwerb von Vorkenntnissen besuchten die Autoren die Vorlesungen des Wahlpflichtfaches \textit{Künstliche Intelligenz}. Eine vorgängige Projektarbeit beschreibt die Grundlagen für dieses Thema. Damit werden nur einzelne Aspekte dieser Bachelor-Thesis dargestellt.

Ziel dieser früheren Projektarbeit war der Aufbau einer semantische Suche für Kinder. Ausgegangen wurde von der Annahme, dass die dazu erforderlichen Technologien bereits existieren. Die Annahme erwies sich als nicht korrekt. Nur gewisse theoretische Grundlagen, wie Ontologien oder Sprachen für deren Modellierung konnten übernommen werden.

\section{Ziele und Abgrenzung}
\label{sec:aufgabe_ziele}
Diese Arbeit besteht aus zwei Teilen. Einerseits dem Tutorial. Dieses zeigt, wie ein Knowledge-Engineer eine Wissensdomäne systematisch modellieren und formalisieren kann. Im zweiten Teil wird eine Ontologie mit Anwendung für ein bestimmtes Problem erstellt. Teil zwei umfasst zusätzlich einen Prototypen einer möglichst benutzerfreundlichen Schnittstelle zur Abfrage der Ontologie.

Eine vollständige Ontologie der Wissensdomäne sollte nicht erstellt werden. Die Ontologie wurde nur soweit entwickelt, dass Abfragen möglich sind, die Mehrwert gegenüber herkömmlichen Suchmaschinen besitzen.
