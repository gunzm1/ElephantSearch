\chapter{Aufgabenstellung}
\label{chap:Aufgabenstellung}

% In diesem Kapitel wir die Aufgabenstellung beschrieben. Erwähnt werden soll die Ausgangslage (was hat es schon gegeben, was ist schon gemacht worden), auf der die Arbeit aufbaut. Im weiteren wird das Ziel beschrieben, welches erreicht werden soll. Daraus leitet sich das in dieser Arbeit gelöste Problem ab.

``Ziel dieser Arbeit ist die Entwicklung und Anwendung eines Systems zur Speicherung (von Daten) in einer Semantischen Datenbank auf der Basis von Apache Stanbol. Dies schliesst die Erstellung einer Domänen-Ontologie mittels RDF/OWL und die Anwendung dieser Ontologie auf ein Anwendungsproblem, wie beispielsweise der Erlernung einer Programmiersprache ein. Exemplarisch soll aufgezeigt werden, wie dabei ein Knowledge Engineer vorgeht, um eine Problemdomäne systematisch zu modellieren und formalisieren. Besondere Bedeutung kommt dabei der Schnittstelle zwischen Mensch und System zu.''~\cite{Aufgabenstellung}

Die obenstehende Aufgabenstellung musste, gegeben durch die gewonnen Erkenntnisse, im Verlaufe der Arbeit geändert werden. So kommt als Werkzeug nicht mehr Apache Stanbol zum Einsatz, sondern Stardog der Firma Clark \& Parsia.
