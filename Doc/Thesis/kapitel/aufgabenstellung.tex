\chapter{Aufgabenstellung}
\label{chap:Aufgabenstellung}

% In diesem Kapitel wird die Aufgabenstellung beschrieben. Erwähnt werden soll die Ausgangslage (was hat es schon gegeben, was ist schon gemacht worden), auf der die Arbeit aufbaut. Im weiteren wird das Ziel beschrieben, welches erreicht werden soll. Daraus leitet sich das in dieser Arbeit gelöste Problem ab.

\section{Motivation}
\label{sec:aufgabe_motivation}
Internet-Suchmaschinen sind heutzutage ein gängiges Mittel um an Wissen im Internet zu gelangen. Deren stetige Weiterentwicklung über das letzte Jahrzehnt macht sie zu einem mächtigen Instrument. Sie werden von vielen Personen zur täglichen Arbeit verwendet.

Suchmaschinen sind heute meist so gehalten, dass in ein Suchfeld Begrifflichkeiten eingegeben werden. Suchmaschinen indexieren Inhalte mittels Stich- und Schlagworten. Deshalb muss der Suchende bereits eine mehr oder minder konkrete Vorstellung von den erwarteten Suchergebnissen haben.

Wenn aber nach Konzepten und Zusammenhängen gesucht werden soll, stossen die heutigen Suchmaschinen schnell an ihre Grenzen.  Sie verfügen über wenig bis kein semantisches Wissen, sie kennen in diesem Sinne aber keine Konzepte. Diese muss der Anwender kennen und entsprechende Stichworte als Eingabe zu einer Suche liefern.

Es existieren zwar Mechanismen zur Speicherung und Darstellung von Konzepten und Zusammenhängen, deren Nutzung ist aber komplex.

Diese Arbeit zielt daher darauf ab eine Art Suchmaschine zur Verfügung zu stellen, welche Wissen über Konzepte und Zusammenhänge hat. Dabei soll sie eine intuitive, einfach zu bedienende Schnittstelle zwischen Mensch und System bieten.

\section{Ausgangslage}
\label{sec:aufgabe_ausgangslage}
Die Grundlagen der Arbeit waren den Autoren teilweise bekannt. Im Rahmen des Studienganges der Autoren an der Berner Fachhochschule für angewandte Wissenschaften existiert aber kein Hauptfach zu dieser Thematik.

Um sich Vorwissen zu dieser anzueignen, besuchten die Autoren das Fach \textit{Künstliche Intelligenz}. Sie verfassten zudem eine Projektarbeit, welche die Grundlagen zu diesem Thema aufgreift. Diese Vorarbeiten betreffen jedoch nur einzelne Aspekte dieser Thesis.

In der vorhergegangenen Projektarbeit war das Ziel eine semantische Suche für Kinder aufzubauen. Dabei wurde davon ausgegangen, dass die Technologien dazu bereits existieren. Dies stelle sich jedoch als nicht gegeben heraus. Einzig gewisse allgemeine theoretische Grundlagen, so z.B. Ontologien oder Sprachen zur Modellierung dieser, konnten übernommen werden.

\section{Ziele und Abgrenzung}
\label{sec:aufgabe_ziele}
Generell besteht diese Arbeit aus zwei Teilen. Einerseits aus einem  Tutorial, welches aufzeigt wie ein Knowledge Engineer eine Wissensdomäne systematisch modelliert und formalisiert. Andererseits aus der Erstellung einer Ontologie und der Anwendung dieser auf ein Anwendungsproblem. Dies umfasst dabei einen Prototypen einer möglichst benutzerfreundlichen Schnittstelle zur Nutzung der Ontologie.

Es ist nicht vorgesehen eine vollständige Ontologie der Wissensdomäne zu erstellen. Diese soll nur soweit entwickelt werden, dass Abfragen möglich sind, welche den Mehrwert gegenüber herkömmlichen Suchmaschinen aufzeigen.
