\chapter{Einleitung}
\label{chap:einleitung}

Das nachfolgende Dokument beschreibt die Anforderungen der Bachelorthesis von Sven Osterwalder und Mira Günzburger. Als Vorarbeit der Bachelorthesis dient die Arbeit, welche im Rahmen des Moduls 7302 "`Projekt 2"' bereits erstellt wurde.

In der Bachelorthesis soll ein Werkzeug für die semantische Suche in einer Wissensdatenbank implementiert werden.

Wie in der Abschlussdokumentation der Projekt 2 Arbeit beschrieben, handelt es sich bei semantischen  Suchmaschinen um "`Werkzeuge, die in der Lage sind, auf Fragen mit Hilfe einer Datenbank oder des Internets Antworten zu generieren. Solche Werkzeuge können insbesondere dann eine sehr wertvolle Unterstützung für den menschlichen Experten sein, wenn unter extremer Zeitnot komplexe Entscheidungen getroffen werden müssen, wie beispielsweise in der medizinischen Diagnostik. Die Firma IBM hat vor nicht allzu langer Zeit für eine Überraschung gesorgt, als sie die Leistungsfähigkeit von „Watson“ im Quiz Jeopardy demonstriert hat. In diesem Quiz, wo schwierige, oft zweideutig formulierte Fragen aus beliebigen Bereichen unter Zeitdruck beantwortet werden müssen, konnte sich Watson überlegen gegenüber zwei bisher sehr erfolgreichen menschlichen Champions durchsetzen.
\cite{projekt2Doc}

Wie im Fazit der Projektarbeit beschrieben, muss der Fokus der Thesis verschoben werden. So soll der Schwerpunkt der Arbeit rein auf der technischen Umsetzung einer semantischen Suche mit Hilfe von Apache Stanbol gesetzt werden. Dies entgegen der ursprünglichen Intention, der Entwicklung eines kindergerechten Frontends.

Nachfolgend werden die einzelnen Aufgaben der Bachelorthesis beschrieben.

% Einträge im Verzeichnis erscheinen lassen ohne hier eine Referenz einzufügen
%\nocite{kopka:band1}
