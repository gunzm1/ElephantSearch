\chapter{Ziel der Thesis}
\label{chap:thesisziel}
Als Endresultat der Thesis soll eine Applikation zur Verfügung stehen, welches es erlaubt eine Frage in deutscher Sprache zur Domäne der Programmierung anhand der Programmiersprache Java zu stellen. Dies kann dank der gegebenen REST-Schnittstelle z.B. direkt per Konsole oder aber per ansprechendem Web-Interface geschehen, welches aber nicht Teil der Thesis ist. Die Applikation soll in der Lage sein mittels der aufgebauten Wissensdatenbank, deren Relationen und schlussendlich Regeln die Frage zu beantworten. Kann eine Frage nicht eindeutig beantwortet werden, sollen zumindest Satzteile (Tokens) extrahiert und der entsprechende Inhalt zu diesen zurückgegeben werden. Eine Antwort ist dabei die Rückgabe einer Entität mit all deren Feldern, welchen dann von dem anfragenden Objekt entsprechend verarbeitet werden kann.
