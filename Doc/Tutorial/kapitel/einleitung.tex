\chapter{Einleitung}
\label{chap:einleitung}
Wir kennen alle den klassischen Ansatz der Wissensabbildung, zum Beispiel in Form von UML, welcher die relationale Datenspeicherung zugrunde liegt. Häufig geschieht dies in enger Verbindung mit der objektorientierten Programmierung. Mit dieser Technik sind Objekteigenschaften und -Verhalten aber schwer abbildbar. Was aber wenn man interessantere Fragestellungen als reine Relationen beantworten möchte? Dies ist mit dem genannten Ansatz praktisch unmöglich.  

Aus diesem Grund möchten wir eine neue, eher unbekanntere Art der Wissensmodellierung vorstellen, welche diese Probleme berücksichtigt. Es handelt sich dabei um die Wissensmodellierung (Knowledge Engineering) welche Teilgebiet der Künstlichen Intelligenz ist. \\
Im folgenden Dokument zeigen wir auf, wie Wissen in einem wissensbasierten System abgebildet werden kann. Wir zeigen dies exemplarisch anhand der Programmiersprache Prolog auf.

Als Hilfe führen wir hier die Eule ein. Das auftreten dieser, ist steht ein Hinweis, dass im folgendem Abschnitt ein praktisch orientierter Tipp folgen wird.



% Einträge im Verzeichnis erscheinen lassen ohne hier eine Referenz einzufügen
%\nocite{kopka:band1}
