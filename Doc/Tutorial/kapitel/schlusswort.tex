\chapter{Schlusswort}
\label{chap:schlusswort}

In der vorliegenden Arbeit haben wir gezeigt, wie ein Knowledge-Engineer bei der Modellierung und Formalisierung einer Problemdomäne vorgehen kann. Am Beispiel Reiseplanung haben wir schrittweise ein Expertensystem aufgebaut. Die Arbeit zeigt formale Aspekte für die Wissensmodellierung auf, wie zum Beispiel verschiedene Sprachen. Aufgrund unserer Erfahrungen geben wir praktische Tipps für die direkte Umsetzung. 


Dieses Expertensystem beinhaltet eine semantische Datenbank, die ihrerseits auf der Basis einer Ontologie beruht.

Wichtige Erkenntniss: Es eignen sich nur Themengebiete, die aufgrund der Ontologie Inferenz und damit Schlussfolgerungen zulassen.

Andernfalls ist der Mehrwert der Ontolgie gegenüber der herkömmlichen Wissensabbildung nicht gegeben.

Während herkömmliche (relationale) Datenbanken nur Beziehungen aufzeigen, liegt der Vorteil der semantischen Datenbank in ihrer Flexibilität und Semantik (Möglichkeit den Relationen eine Bedeutung zu geben).

Nutzt eine Applikation eine semantische Datenbank als Datenmodell, erfordern Anpassungen (Modellierungen) des Datenmodelles keine Programmänderungen --- bei geschickter Programmierung.\\
Modellierungen sind z.B. das Hinzufügen, Bearbeiten oder Löschen von Entitäten (Klassen, Individuen, Relationen oder Eigenschaften).

Im Gegensatz hierzu benötigen Änderungen in relationalen Datenbanken meistens sehr aufwendige Programmänderungen.
