\chapter{Schlusswort}
\label{chap:schlusswort}

In der vorliegenden Arbeit zeigen wir auf, wie bei der Modellierung sowie Formalisierung der Wissensmodellierung mittels Ontologien vorgegangen werden kann. Dabei bieten wir verschiedene Sichten auf das Vorgehen. 

Es war uns bei der Arbeit wichtig, die theoretischen Grundlagen fundiert und bis zu einem gewissen Detailgrad aufzuzeigen. \\
Am Beispiel Reiseplanung haben wir schrittweise eine Ontologie aufgebaut und führen den Leser so durch den gesamten Prozess der Modellierung. Aufgrund unserer Erfahrung geben wir praktische Tipps für die direkte Umsetzung. Zusätzlich zeigen wir anhand von Beispielen auf, wie die einzelnen Themen der Arbeit in der Praxis umgesetzt werden können.

Ontologien bilden die Grundlage für eine semantische Datenbank, welche wiederum die Wissensbasis für ein Expertensystem ist. Ontologien beinhalten sowohl Fakten, als auch Regeln. Die Fakten werden mittels einer Ontologiesprache wie zum Beispiel OWL beschrieben.Regeln werden in der Regelsprache SWRL abgebildet.\\
Die Inferenzmaschine (Reasoner) ist eine weitere Komponente des Expertensystems. Ein Reasoner ist fähig mittels Inferenz und Resolution Schlüsse zu ziehen und neues Wissen zu generieren.\\
Diese Eigenschaft führt zu dem Mehrwert dieser Form der Wissensmodellierung. Die Daten werden so mit einer gewissen Intelligenz mittels Schlussfolgerungen angereichert.\\
Während herkömmliche (relationale) Datenbanken nur Beziehungen aufzeigen, liegt der Vorteil der semantischen Datenbank in ihrer Flexibilität und Semantik (Möglichkeit den Relationen eine Bedeutung zu geben). Es werden also nicht nur Informationen, sondern auch Wissen abgebildet.

Eine Ontologie wird immer auf eine spezifische Problemdomäne angewendet. Eine wichtige Erkenntnis war für uns, dass sich nicht jedes Themengebiet als Problemdomäne eignet. So macht eine Abbildung einer solchen Domäne mittels Ontologien nur dann Sinn, wenn ein Experte benötigt wird um das vorhandene Wissen zu interpretieren.
Theoretische Gebiete auf hoher Abstraktionsebene sind nicht geeignet.\\
Zur Modellierung bieten sich verschiedene Hilfsmittel an. Unserer Erfahrung nach sind semantische Netze eine sehr nützliche und übersichtliche Art der grafischen Darstellung.

Nutzt eine Applikation eine semantische Datenbank als Datenmodell, erfordern Anpassungen (Modellierungen) des Datenmodelles keine Programmänderungen --- bei geschickter Programmierung.\\
Modellierungen sind z.B.\ das Hinzufügen, Bearbeiten oder Löschen von Entitäten (Klassen, Individuen, Relationen oder Eigenschaften).
Im Gegensatz hierzu benötigen Änderungen in relationalen Datenbanken meistens sehr aufwendige Programmänderungen.




