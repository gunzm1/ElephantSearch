\chapter{Schlusswort}
\label{chap:schlusswort}

In der vorliegenden Arbeit zeigen wir auf wie bei der Modellierung sowie Formalisierung der Wissensmodellierung mittels Ontologien vorgegangen werden kann. Dabei haben wir diese Arbeit aus verschiedenen Sichten beleuchtet. 

Es war es uns wichtig, die theoretischen Grundlagen fundamental und einigermassen detailliert aufzuzeigen. \\
Am Beispiel Reiseplanung haben wir schrittweise eine Ontologie aufgebaut. Damit führen wir den Leser durch die gesamte Modellierung. Der Fokus liegt dabei einerseits bei praktischen Tipps, andererseits bei konkreten Beispielen. 

Ontologien bilden die Grundlage für eine Semantische Datenbank, welche wiederum die Wissensbasis für ein Expertensystem darstellen. Die Ontologien beinhalten sowohl die Fakten welche mittels einer Ontologiesprache wie OWL beschrieben sind, als auch die Regeln. Diese werden in SWRL abgebildet. Ein Expertensystem hat als weitere Komponente den Reasoner (Inferenzmaschine). Der Reasoner ist fähig mittels Inferenz und Resolution Schlüsse zu ziehen und neues Wissen zu generieren.\\
Diese Eigenschaft führt zu dem Mehrwert dieser Form der Wissensmodellierung. Die Daten werden so mit einer gewissen Intelligenz.\\
Während herkömmliche (relationale) Datenbanken nur Beziehungen aufzeigen, liegt der Vorteil der semantischen Datenbank in ihrer Flexibilität und Semantik (Möglichkeit den Relationen eine Bedeutung zu geben). Es werden also nicht nur Informationen, sondern auch Wissen abgebildet.

Eine Ontologie wird immer auf eine Problemdomäne angewendet. Als wichtige Erkenntnis haben wir festgestellt, dass sich nicht jedes Themengebiet als Domäne eignet. So macht eine Abbildung mittels Ontologien nur dann Sinn, wenn eine Domäne aus Fakten besteht. Theoretische Gebiete auf hoher Abstraktionsebene sind nicht sehr geeignet da ohne Fakten keine Schlussfolgerungen gezogen werden können.\\
Zur Modellierung bieten sich verschiedene Hilfsmittel an. In Unseren Augen ist eine sehr nützliche und übersichtliche Art der grafischen Darstellung, die Abbildung in Form eines semantischen Netzes.

Nutzt eine Applikation eine semantische Datenbank als Datenmodell, erfordern Anpassungen (Modellierungen) des Datenmodelles keine Programmänderungen --- bei geschickter Programmierung.\\
Modellierungen sind z.B. das Hinzufügen, Bearbeiten oder Löschen von Entitäten (Klassen, Individuen, Relationen oder Eigenschaften).
Im Gegensatz hierzu benötigen Änderungen in relationalen Datenbanken meistens sehr aufwendige Programmänderungen.




