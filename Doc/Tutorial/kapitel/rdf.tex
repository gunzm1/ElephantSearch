\chapter{RDF}
\label{chap:rdf}
Jetzt haben wir einen Überblick über die verschiedenen Methoden und Herangehensweisen zur Generierung von Knowledge Engineering. Auch die wichtigsten theoretischen Grundlagen haben wir mit den vorigen Kapiteln abgedeckt. Im Kapitel \nameref{chap:ontologien} haben wir gelernt, dass für die Darstellung von Ontologien verschiedene Sprachen entwickelt wurden. In Diesem Beispiel wird eine der bekanntesten Ontologie Sprachen OWL (Web Ontology Language) verwendet. OWL basiert auf der RDF Syntax. Folglich scheint es uns wichtig diese zwei Sprachen kurz vorzustellen. 

Das RDF Kapitel basiert auf der Spezifikation von w3 \cite{w3rdf}.\\
Das "`Resource Description Framework"' RDF ist ein Framework um Informationen aus Ressourcen zu formulieren. Ressourcen können Dokumente, Leute, Objekte aber auch abstrakter Inhalt darstellen. Mit RDF können Informationen im Web von Anwendungen verarbeitet werden, anstatt sie nur anzuzeigen. RDF bietet ein gemeinsames Framework um die Informationen zwischen Anwendungen auszutauschen ohne Bedeutung zu verlieren. 

RDF ist die Grundlage für Sematic Web, welches seine Flexibilität ausnützt. Alle Daten im Semantic Web werden in RDF abgebildet. Wichtig dabei ist, dass RDF es ermöglicht Daten zu verknüpfen. Dies führt dazu, dass zu einer Ressource mehr Informationen zusammen getragen werden.\cite{cambSemRDF}

%\subsection{Einsatzgebiete}
%\label{sec:logischeProgrammierung_herkunft_einsatz}
\section{RDF Data Model}
\label{sec:rdf_rdf_dataModel}
Die Informationen werden in RDF als Aussagen abgebildet. Der Aufbau dieser  ist immer gleich und weist die folgende Struktur eines Tripels auf:

\noindent\hspace*{15mm}<Subjekt><Prädikat><Objekt>

Eine RDF Aussage bildet eine Beziehung zwischen zwei Ressourcen (Entitäten) ab. 
Subjekt und Objekt stellen die zwei Ressourcen dar. Das Prädikat repräsentiert die Beziehung zwischen den zwei Ressourcen Subjekt und Objekt. Die Beziehung wird in RDF als Property abgebildet. 
 
\noindent\hspace*{15mm} <Eine Programmiersprache> <hat> <ein Programmierparadigma>


Eine Entität kann in mehreren Tripeln referenziert werden. Es ist zudem möglich eine Ressource in einer Aussage als Objekt und in einer Anderen als Subjekt zu verwenden. dies ermöglicht es, Verbindungen zwischen mehreren Tripeln herzustellen. Dies ist ein wichtiger Teil von RDF.

Tripel werden in sogenannte RDF Graphen abgebildet. Diese bestehen aus Knoten und Pfeilen. die Subjekte und die Objekte werden als Knoten, die Prädikate als Pfeile dargestellt. Genaueres dazu findet sich im Kapitel \nameref{chap:graph_data}



TODO Bildli von Unserer Ontologie

Es gibt drei Typen von RDF Daten, welche in Tripeln auftreten, IRIs (oder Ressource Knoten) Literale und Blank (Leere) Knoten.

\subsection{IRIs (International Resource Identifier)}
\label{sec:rdf_rdf_dataModel_iris}
Wie der Name schon sagt, stellt ein IRI  eine Ressource dar. Dabei handelt es sich um einen globalen Identifier, IRIS können also von verschiedenen Nutzern wiederverwendet werden. Es gibt verschieden Formen von IRIs, so zum Beispiel die URLs welche als Web Adresse verwendet werden. Eine Andere Form der IRI bietet eine Kennung für eine Ressource ohne den Standort oder den Zugriff preiszugeben.% IRIs sind im RFC 3987 spezifiziert.

\noindent\hspace*{15mm}IRIs kann in allen drei Positionen eines Tripels auftreten.

\subsection{Literale Knoten}
\label{sec:rdf_rdf_dataModel_literal}
Der Begriff Literal wird als Synonym für Wert verwendet. Es handelt sich bei Literalen also um Basiswerte die nicht IRIS sind. Literale können Strings, Datumswerte oder auch Nummern sein. Um die Werte richtig zu interpretieren, haben Literale einen Datentyp zugeordnet. Ein String kann zusätzlich eine Sprache zugewiesen haben.

Literale können in einem Tripel nur als Objekt verwendet werden.

\subsection{Blank Nodes}
\label{sec:rdf_rdf_dataModel_blankNodes}
Ein leerer Knoten stellt eine Ressource ohne URI dar. Der Vorteil dieser Knoten ist, dass es keinen globalen Identifier braucht. Leere Knoten können mit einer einfachen Variable in der Algebra verglichen werden. Sie bilden ein Objekt ab, wobei der Wert zweitrangig ist. 

\noindent\hspace*{15mm} Leere Knoten können in einem Tripel in der Subjekt oder der Objekt Position stehen.


\section{Multiple Graphs}
\label{sec:owlRdf_rdf_dataModel_multipleGraphs}

Eine der neuesten Erweiterung von RDF sind Multipe Graphen. Diese wurden eingeführt um Teilmengen einer Tripelsammlungen zu definieren. Dieser Mechanismus stammt ursprünglich von der Abfragesprache SPARQL.\ref{chap:sparql}
Es gibt sogenannte named und unnamed Graphen. 
Bei einem Unamend Graph enthalten die Tripel die gesammte URI.
\begin{lstlisting}[caption={Named graph}]
	<http://example.org/bob> <is published by> <http://example.org>.
\end{lstlisting}
Bei named Graphen wird ein identifier geschaffen auf den Referenziert wird. 
\begin{lstlisting}[caption={Named graph}]
Identifier: 
		http://example.org/bob
Graph:
		<Bob> <is a> <person>.
		<Bob> <is a friend of> <Alice>.
\end{lstlisting}
Multipe Graphen eines RDF Dokuments stellen eine Datenmenge dar. Sie bestehen standartmässig aus einem unnamed (default) graphen und mehreren named Graphen.


\section{RDF Vokabular}
\label{sec:rdf_rdf_voca}
RDF wird typischerweise in Kombination mit Vokabular und Konventionen verwendet, welche Semantische Daten zu Ressourcen zur Verfügung stellen.

Um das Vokabular von RDF zu zu definieren wird die RDF Schema Sprache unterstützt. Diese ermöglicht semantische Eigenschaften der RDF Daten zu definieren. Es kann damit festgelegt werden, welche Ressourcen an welcher Position verwendet werden.

so verwendet RDF zum Beispiel die Bezeichnung der Klasse um zu kategorisieren. Die Beziehungen zwischen Instanzen werden in Propertys abgebildet. Mit RDF können auch Hierarchien im Bereich der Klassen aber auch der Eigenschaften gebildet werden. Ausserdem können auf Objekten und Subjekten Typeinschränkungen vorgenommen werden.\footnote{http://www.w3.org/TR/2014/NOTE-rdf11-primer-20140624/\#section-rdfa\cite{w3rdf}}

\section{RDF Formen}
\label{sec:rdf_rdf_formen}

Es gibt verschiedene Formen um RDF niederzuschreiben. Diese führen alle zu den gleichen Tripels. Sie haben unterschiedliche Einsatzgebiete. So gibt es zum Beispiel die N-Tripel oder die Turtle Schreibweise. Im folgenden Kapitel soll aber das XML/RDF Format kurz vorgestellt werden, da wir es in unserem Beispiel verwenden.

\subsection{XML/RDF}
\label{sec:rdf_rdf_formen_xmlRdf}
Bei XML/RDF handelt es sich um eine Schreibweise von RDF, welche die XML Syntax verwendet um  RDF Graphen abzubilden. Als RDF in den 1990er Jahren entwickelt wurde, war dies die einzige Schreibweise dafür. 

In RDF/XML sind Tripel in einem XML Element rdf:RDF spezifizert.  Das Element rdf:description wird verwendet um ein Tripel zu definieren, welches als Subjekt, die im about Attribut definierte IRI Spezifikation, hat. Ein Descriptionelement kann Unterelemente beinhalten. Der Name des Subelements is ein IRI, welches in der rdf\_property rdf:type abgebildet ist. Dabei repräsentiert jedes Subelement ein Tripel. Handelt es sich beim Objekt eines Tripels auch um ein IRI, hat das Unterelement keinen Inhalt. Der Objekt IRI Knoten wird durch ein rdf:resource Attribut beschrieben.

Wenn das Objekt eines Tripels ein Literal ist, ist der Literalwert als Inhalt des Elements angegeben. Datentypen sind auch als Attribute eines Elementes angegeben. 

Hierzu ein Beispiel:(TODO Eigenes Beispiel)\\
\begin{lstlisting}[caption={Beispiel RDF Elemente\protect\footnotemark}]
01    <?xml version="1.0" encoding="utf-8"?>
02    <rdf:RDF
03             xmlns:dcterms="http://purl.org/dc/terms/"
04             xmlns:foaf="http://xmlns.com/foaf/0.1/"
05             xmlns:rdf="http://www.w3.org/1999/02/22-rdf-syntax-ns\#"
06             xmlns:schema="http://schema.org/">
07       <rdf:Description rdf:about="http://example.org/bob\#me">
08          <rdf:type rdf:resource="http://xmlns.com/foaf/0.1/Person"/>
09          <schema:birthDate rdf:datatype="http://www.w3.org/2001/XMLSchema\#date">1990-07-04</schema:birthDate>
10          <foaf:knows rdf:resource="http://example.org/alice\#me"/>
11          <foaf:topic\_interest rdf:resource="http://www.wikidata.org/entity/Q12418"/>
12       </rdf:Description>
13       <rdf:Description rdf:about="http://www.wikidata.org/entity/Q12418">
14          <dcterms:title>Mona Lisa</dcterms:title>
15          <dcterms:creator rdf:resource="http://dbpedia.org/resource/Leonardo\_da\_Vinci"/>
16       </rdf:Description>
17       <rdf:Description rdf:about="http://data.europeana.eu/item/04802/243FA8618938F4117025F17A8B813C5F9AA4D619">
18          <dcterms:subject rdf:resource="http://www.wikidata.org/entity/Q12418"/>
19       </rdf:Description>
20    </rdf:RDF>
\end{lstlisting}
\footnotetext{Beispiel aus \cite{w3rdf}}
