\chapter{Expertensysteme}
\label{chap:experten_systeme}

Bei Expertensystemen handelt es sich um Systeme zur Wissenrepräsentation, welche, in einem sehr eingeschränkten (Teil-) Gebiet, die Leistung eines menschlichen Experten erbringen oder diese sogar übertreffen. Die Entwicklung von Expertensystemen und deren Erfolg führten zu einem standardisierten Prozess der Wissensdarstellung und Ontologien, was schlussendlich die Entwicklung von Expertensystemen in neuen, unerforschrten Themengebieten erheblich vereinfachte.~\cite[S. 257]{russel}

\begin{figure}[htbp]
\centering \rotatebox{0}{\scalebox{0.5}[0.5]{\includegraphics{bilder/aufbau_expertensysteme.png}}}
\caption{Aufbau eines Expertensystems.\label{fig:aufbau_expertensysteme}\protect\footnotemark}
\end{figure}
\footnotetext{\cite[S. 23]{laemmel}}

\section{Komponenten}
\label{sec:experten_systeme_komponenten}
Um ein Expertensystem aufbauen zu können, muss zuerst einmal das Wissen über einen Problembereich formalisiert werden. Die dazu benötigten Komponenten sind:
\begin{itemize}
    \item Eine Wissensdatenbank, welche die Fakten des Problembereiches enthält
    \item Eine formale Sprache zur Wissensrepräsentation
    \item Einen Verarbeitungsmechanismus zum automatischen Ziehen von Schlüssen
\end{itemize}

\section{Problemlösung}
\label{sec:experten_systeme_problemloesung}
Um ein Problem zu lösen, wird typischerweise in den folgenden Schritten vorgegangen:
\begin{itemize}
    \item Charakterisierung der Problemdomäne
    \item Symbolische Repräsentation der Objekte
    \item Eingabe des Wissens in den Computer
    \item Stellen von Fragen
    \item Interpretieren der Antworten
\end{itemize}

Für die symbolische Repräsentation der Objekte ist es notwendig eine geeignete Sprache zu wählen. Dies können beispielsweise mathematische Relationen, Logik oder auch eine Programmiersprache sein. In der Informatik wird üblicherweise das Wissen über ein Problem direkt in dem Lösungsalgorithmus programmiert, bei der künstlichen Intelligenz jedoch wird das Wissen getrennt von der Verarbeitungskomponente dargestellt. Dies hat den Vorteil, dass die Wissensbasis jederzeit ausgewechselt werden kann, die Verarbeitungskomponente jedoch bestehen bleibt. Ein Programm kann somit also für unterschiedliche Anwendungen verwendet werden. (vgl.~\cite[S. 28 - 30]{laemmel})

Eine Problemlösung geht immer von dem vorhandenen, expliziten Wissen --- z.B. in Form von Fakten --- aus. Aus dem expliziten Wissen kann implizites Wissen gewonnen, also Aussagen impliziert werden. Die Aufgabe der Verarbeitungskomponente ist es nun das implizite Wissen abzuleiten. (vgl.~\cite[S. 30 - 31]{laemmel}) Dabei muss die Sprache zur Wissensrepräsentation sowie die Verarbeitungskomponente gewissen Kriterien genügen, Details siehe~\cite[S. 31]{laemmel}.

\section{Wissensarten}
\label{sec:experten_systeme_wissensarten}
Der Mensch nutzt mehrere Arten um Wissen abzubilden:
\begin{itemize}
    \item Relationales Wissen
    \item Vererbung von Eigenschaften
    \item Prozedurales Wissen
    \item Logisches Wissen
\end{itemize}
\label{itm:wissensarten}

\subsection{Relationales Wissen}
\label{subsec:relationales_wissen}
Relationales Wissen wiederspiegelt einfache Beziehungen zwischen Objekten. Ein Nachteil am relationalen Wissen ist, dass nur Fakten, aber keine logischen Abhängigkeiten abgebildet werden können.

\subsection{Vererbung von Eigenschaften}
\label{subsec:vererbung_eigenschaft}
Bei der Vererbung von Eigenschaften geht es darum, dass Eigenschaften einer Oberklasse an eine Unterklasse weitervererbt werden. Die logischen Programmiersprachen verwenden beispielsweise Regeln. Definiert man nun, dass Prolog eine solche logische Programmiersprache ist, ist es für den Menschen intuitiv klar, dass auch Prolog auch Regeln verwendet wird. Die Unterklasse Prolog erbt also die Eigenschaft der Verwendung von Regeln von deren Oberklasse, der logischen Programmiersprache.

\subsection{Prozedurales Wissen}
\label{subsec:prozedurales_wissen}
Bei prozeduralen Wissen handelt es sich um Wissen, welches in bestimmten Situationen bestimmte Aktionen vorschreibt. Dies kann auch als Folge von Aktionen verstanden werden. So zum Beispiel das Aufschliessen einer Türe: Man steckt den (passenden) Schlüssel in da Schlüsselloch, dreht diesen, entsichert das Schloss, drückt die Türfalle nach unten und öffent schliesslich die Türe.

\subsection{Logisches Wissen}
\label{subsec:logisches_wissen}
Bei logischem Wissen geht es im Grunde genommen um eine logische Implikation. Aus $A$ folgt $B$ bzw. $A \to B$, was so viel heisst wie ``Wenn $A$ gilt, kann geschlossen werden, dass auch $B$ gilt.''. Sei Ereignis $A$ das Öffnen einer Türe und $B$ die Tatsache, dass die Türe offen ist, kann gesagt werden, dass aus $A$, also dem Öffnen der Türe, $B$ folgt, nämlich dass die Türe danach geöffnet ist.

Formalisiert die unter~\ref{itm:wissensarten} genannten Arten des Wissens, so gelangt man zu den folgenden Wissenrepräsentationsformalismen:
\begin{itemize}
    \item Logik
        \begin{itemize}
            \item Aussagenlogik
            \item Prädkatenlogik erster Stufe
        \end{itemize}
    \item Semantische Netze und Frames
    \item Regelbasierte Sprachen
\end{itemize}

