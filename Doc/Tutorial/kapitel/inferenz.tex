\chapter{Inferenz}
\label{chap:inferenz}

Grundsätzlich geht es bei Inferenz um den Prozess Schlussfolgerungen mithilfe von Resolution --- aus dem Lateinischen ``resolutio'', zu Detsch ``Auflösung'', ein Algorithmus von J. A. Robinson für Prädikatenlogik erster Stufe~\cite[S. 18]{russel} --- zu ziehen. --- [Logische Inferenz erklären] ---. Die logische Inferenz ist ein Prozess der Inferenz und Resolution, welcher die Folgebeziehungen zwischen Sätzen zum Ausdruck bringt.~\cite[S. 163]{russel}.

Eine Aussage ist dann wahr, wenn sie unter allen möglichen Interpretationen und Gegebenheiten wahr ist. So handelt es sich zum Beispiel bei der Aussage

\noindent\hspace*{12mm}\textit{``Schrödingers Katze lebt noch oder ist bereits tot''}

um solch eine Aussage. Die Aussage ist wahr, da sie in beiden Fällen --- wenn die Katze noch lebt oder bereits gestorben ist --- wahr ist.

Hingegen ist die Aussage

\noindent\hspace*{12mm}\textit{``Du hast entweder einen Bruder oder eine Schwester''}

nicht per se wahr. Sie ist nur unter der Annahme, dass die Person nur über einen Bruder oder eine Schwester verfügt, wahr. So müsste die Aussage korrekterweise

\noindent\hspace*{12mm}\textit{``Falls Du nicht einen Bruder und eine Schwester hast, hast Du entweder einen Bruder oder eine Schwester''}

lauten.

Gültige Aussagen werden in der Literatur auch als \textit{analytische Aussagen} oder \textit{Tautologien} bezeichnet.~\cite[S. 164]{russel}

\section{Erfüllbare und nicht erfüllbare Aussagen}
\label{sec:erfüllbar-nicht-erfüllbar}

Abschnitt nach~\cite[S. 164]{russel}.
Eine Aussage gilt dann als \textit{erfüllbar}, wenn unter mindestens einer Gegebenheit mindestens eine Interpretation existiert, welche zutrifft. Ist dies nicht der Fall, so gilt eine Aussage als \textit{nicht erfüllbar}. So gelten widersprüchliche 
Aussagen auch als nicht erfüllbar --- sofern die Widersprüchlickeit nicht von der Bedeutung der Symbole abhängt. Ein widersprüchlicher Satz ist zum Beispiel

    \noindent\hspace*{12mm}\textit{``Morgen habe ich frei und morgen muss ich arbeiten''}.

\section{Inferenz in Computern}
\label{sec:inferenz-in-computer}

Das grundsätzliche Problem bei Computern im Bezug auf Inferenz ist, dass ein Computer keine Interpretation vornehmen kann und nnichts über die (Um-)Welt weiss bzw. was in seiner Wissensdatenbank gespeichert ist.

Angenommen, man möchte einen Computer fragen

    \noindent\hspace*{12mm}\textit{``Kann ein Vogel fliegen?''}

so weiss der Computer weder was ein Vogel ist, noch kennt er das Konzept des Fliegens. Das Einzige, was er tun kann, ist, in der Wissensdatenbank nach 

    \noindent\hspace*{12mm}\textit{``Ein Vogel kann fliegen''}

zu suchen. Findet der Computer diese Aussage in der Wissensdatenbank, so spielt es keine Rolle, dass er das Konzept des Fliegens oder Vögel nicht kennt. Die Schlussfolgerung, dass ein Vogel fliegen kann, trifft unter allen Gegebenheiten und Interpretationen zu, welche für die Wissensdatenbank zutreffen.\footnote{Überarbeitung nötig!}

Zusammengefasst kann gesagt werden, dass die formale Inferenz in der Lage ist, gültige Schlussfolgerungen zu ziehen, auch wenn der Computer die Interpretation(en) des Anwenders nicht kennt. Der Computer zieht immer logisch gültige Schlüsse, unabhängig von der (menschlichen) Interpretation. Da der Mensch in der Regel die Interpretation kennt, erscheinen die Schlüsse dem Menschen logisch.

% Russel \& Norvig
% Einträge im Verzeichnis erscheinen lassen ohne hier eine Referenz einzufügen
%\nocite{kopka:band1}
