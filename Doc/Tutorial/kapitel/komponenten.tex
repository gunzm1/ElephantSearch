\chapter{Komponenten}
\label{chap:komponenten}

Bei den Recherchen der Projektarbeit hat sich die Erkenntnis ergeben, das eine erfolgreiche Verarbeitung von Anfragen die folgenden Komponenten benötigt:
\begin{itemize}
	\item Abbildung der Umwelt mittels Wissensdatenbank
	\item Definieren von Regeln zur Ableitung von Schlüssen mittels Logik
	\item Interne und externe Schnittstellen zur Kommunikation
\end{itemize}

Die oben genannten Komponenten werden bereits durch die in der Projektarbeit evaluierten Lösung --- Apache Stanbol --- zur Verfügung gestellt. Allerdings hat sich gezeigt, dass diese keine dem Menschen leicht verständliche Möglichkeit bietet, um eine Fragestellung  in ein für den Menschen logisches Resultat zu überführen.

So kann zum Beispiel die Frage `Welches ist die Hauptstadt der Schweiz?` nicht einfach so beantwortet werden. Die Lösung bietet Extraktion von Entitäten, deren Eigenschaften und Relationen sowie Abfragen dieser mittels einer speziellen Sprache (SPARQL). Die Überführung der extrahierten Entitäten in eine Abfrage, welche für den Menschen ein sinnvolles Resultat zurückliefert, ist nicht gegeben.

\section{Architektur}
\label{sec:architektur}
Um die verschiedenen Teile zusammen zu nutzen, bietet Apache Stanbol eine frei konfigurierbare Verkettung dieser an. Dies geschieht mittels einer so genannten Enhancement-Chain. Konkret heisst dies, dass eine beliebige Eingabe dieser Kette übergeben werden kann, worauf dann die erste Softwarekomponente der Kette die Eingabe verarbeitet und das Resultat an die nächste Komponente weiterreicht. Diese Vorgang wird durch sämtliche Komponenten der Kette fortgeführt bis schlussendlich das Endresultat an die anfragende Entität zurückgegeben wird.

\section{Abbildung der Umwelt mittels Wissensdatenbank}
\label{sec:architektur_wissensdatenbank}

\subsection{Objekte abbilden}
\label{subsec:architektur_wissensdatenbank_Objekte}
Die Abbildung der Umwelt geschieht in Apache Stanbol mittels dem so genannten Entity Hub.
Dieser stellt Informationen zu Entitäten und Objekten einer spezifischen Wissensdomäne zur Verfügung. Die Beziehungen werden in Apache Stanbol in Form von Relationen zwischen den Entitäten abgebildet, analog dazu werden die Eigenschaften als Attribute erfasst.

Die konkrete Arbeit mit dem Entity Hub besteht also darin Objekte, der für die Arbeit gewählten Domäne, als Entitäten abzubilden.

\subsection{Definieren von Regeln zur Ableitung von Schlüssen mittels Logik}
\label{sec:architektur_regeln}
Um nun aus der Wissensdatenbank Schlüsse ziehen zu können, werden Regeln benötigt. Regeln werden verwendet um mittels Bedingungen auf weitere Eigenschaften schliessen zu können.
Apache Stanbol unterstützt auf Prädikatenlogik basierende Regeln, welche innerhalb der Rule Store Komponente als Rezepte gespeichert werden. Diese sind nichts anderes als eine Zusammenfassung von Regeln, welche eine ähnliche Objektkategorie betreffen.

\section{Spracherkennung}
\label{sec:architektur_spracherkennung}
Um einen Eingabesatz auszuwerten, muss dieser erst als solcher erkannt werden. Konkret müssen also, die einzelnen Wörter als Tokens identifiziert und einer Sprachkategorie zugeorndet werden. Dies geschieht mittels der Spracheerkennungskomponente OpenNLP (open natural language processing). Diese Spracherkennung wird für die Englische Sprache von Stanbol schon weitgehend angeboten.


\section{Interne und Externe Schnittstellen zur Kommunikation}
\label{sec:architektur_schnittstellen}
Um die oben genannten Komponenten ansprechen und nutzen zu können, sind Schnittstellen zur Kommunikation unumgänglich.

\subsection{Interne Kommunikation}
\label{sec:architektur_schnittstellen_intern}
Intern nutzt Apache Stanbol, wie bereits beschrieben, die so genannte Enhancement Chain um von einem gegebenen Input, mittels verschiedenen Komponenten, zu einem Output zu gelangen~\cite{stanbol:enhancementchain}. Die Enhancement Chain basiert auf einer Graph-Struktur, welche sie zur Kommunikation nutzt.

\begin{figure}[H]
	\centering
    \includegraphics[scale=0.4]{bilder/enhancementstructure.png}
	\caption{Apache Stanbol Enhancement Chain\protect\footnotemark}
\label{fig:kommunikationKomponenten}
\end{figure}
\footnotetext{\cite{stanbol:enhancementgraph}}

\newpage

\subsection{Externe Kommunikation}
\label{sec:architektur_schnittstellen_extern}
Jede Komponente von Apache Stanbol, so z.B. auch die Enhancement Chain sowie deren Einzelkomponenten, verfügt über ein REST-Interface. Dies dient zur Kommunikation gegen aussen. Ein schematischer Ablauf der Kommunikation wird in Abbildung~\ref{fig:kommunikationKomponenten} grob dargestellt.

\begin{figure}[H]
	\centering
	\includegraphics[scale=0.4]{bilder/software_komponenten.png}
	\caption{Kommunikation im Überblick\protect\footnotemark}
\label{fig:kommunikationKomponenten}
\end{figure}
\footnotetext{Eigene Darstellung mittels Libre Office Writer}
