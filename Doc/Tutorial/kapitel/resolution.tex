\chapter{Resolution}
\label{chap:resolution}

Um das Konzept der Resolution aufzeigen zu können, ist es erst notwendig Logik einzuführen.

\section{Logik}
\label{sec:logik}

Eine Logik besteht aus den folgenden Elementen:

\begin{itemize}
    \item{Ein formales System, welches Zustände beschreibt, bestehend aus}
    \begin{itemize}
        \item{Syntax einer Sprache

        welche beschreibt, wie Sätze erstellt werden können}
        \item{Semantik einer Sprache

        welche die Bedeutung beschreibt, also wie Sätze im Zusammenhang mit Zuständen stehen}
    \end{itemize}
    \item{Beweistheorie

        welche aus einem Satz von Regeln besteht und es somit erlaubt eine (logische) Ableitung zwischen Sätzen vorzunehmen
    }
\end{itemize}

\subsection{Prädikatenlogik erster Stufe}
\label{subsec:prädikatenlogik}

Im Gegensatz zur Aussagenlogik, welche hier nicht explizit behandelt wird, bietet die Prädikatenlogik erster Stufe eine stärkere Bindung zu der Ontologie, also der Speicherung von Informationen, welche über logische Relationen verfügen.~\cite[S. 185]{russel}

Die Prädikatenlogik erster Stufe bietet somit folgende Elemente:
\begin{itemize}
    \item{\textbf{\textit{Objekte}}

        Individuelle Entitäten
    }
    \item{\textbf{\textit{Eigenschaften}}

        welche die Entitäten voneinander unterscheiden
    }
    \item{\textbf{\textit{Relationen}}

        welche die Beziehungen zwischen den Entitäten beschreiben
    }
    \item{\textbf{\textit{Funktionen}}

        eine spezielle Form einer Relation, für welche nur eine bestimmte Ausgabe für eine bestimmte Eingabe existiert
    }
\end{itemize}

\newpage

Um dies anhand eines Beispiels zu verdeutlichen:

    \noindent\hspace*{12mm}\textit{``Eine Programm verwendet eine Programmiersprache''}

Angewendet auf die Prädikatenlogik erster Stufe ergibt sich Folgendes:

\begin{itemize}
    \item{Objekt}
    \begin{itemize}
        \item{Programm}
        \item{Programmiersprache}
    \end{itemize}
    \item{Relation}
    \begin{itemize}
        \item{verwendet}
    \end{itemize}
\end{itemize}

\subsubsection{Syntax und Semantik}
\label{subsubsec:syntax-und-semantik}

\textbf{Terme} \textit{Konstanten, Variablen und Funktionen}

In der Prädikatenlogik existieren sowohl Sätze, als auch Terme. Sätze repräsentieren einen Fakt, Terme räpresentieren Objekte. Dabei bestehen Terme aus Konstanten, Variablen und Funktionen, Sätze hingegen aus Quantoren und Prädikaten.


\textbf{Konstanten} \textit{A, B, C, Programmiersprache etc.}

Konstanten repräsentieren Objekte, wobei eine Konstante genau ein Objekt repräsentiert, aber nicht alle Objekte eine Konstante haben müssen. Es ist auch möglich, dass ein Objekt mehrere Konstanten haben.


\textbf{Prädikate} \textit{Programmiersprache, Programmierparadigma etc.}

Ein Prädikat definiert eine Relation im Modell (der Welt). So steht z.B. Programmiersprache dafür, dass ein Programm eine Programmiersprache verwendet. Es handelt sich dabei also um eine Relation zwischen einem Paar von Objekten. Diese Relation wird in der Regel durch \textit{Tupel} definiert.


\textbf{Tupel} \textit{\{<Programm1, Prolog>, <Programm2, Java>\}}

Bei Tupeln handelt es sich um eine Kollektion von Objekten mit einer definierten Ordnung.


\textbf{Funktionen} \textit{VerwendetProgrammiersprache, VerwendetProgrammierparadigma}

Es existieren Relationen, welche funktional sind. Dies bedeutet, dass ein gegebenes Objekt mit \textit{genau einem} anderen Objekt in Relation steht. Z.B. verwendet ein Programm nur genau eine Programmiersprache, also:

\lstset{language=Prolog}
\begin{lstlisting}
    VerwendetProgrammiersprache (Programm1, Prolog)
\end{lstlisting}


% Russel \& Norvig

% Einträge im Verzeichnis erscheinen lassen ohne hier eine Referenz einzufügen
%\nocite{kopka:band1}
