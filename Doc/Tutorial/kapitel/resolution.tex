\chapter{Resolution}
\label{chap:resolution}

Um das Konzept der Resolution aufzeigen zu können, ist es erst notwendig Logik einzuführen.

\section{Logik}
\label{sec:logik}

Eine Logik besteht aus den folgenden Elementen:

\begin{itemize}
    \item{Ein formales System, welches Zustände beschreibt, bestehend aus}
    \begin{itemize}
        \item{Syntax einer Sprache

        welche beschreibt, wie Sätze erstellt werden können}
        \item{Semantik einer Sprache

        welche die Bedeutung beschreibt, also wie Sätze im Zusammenhang mit Zuständen stehen}
    \end{itemize}
    \item{Beweistheorie

        welche aus einem Satz von Regeln besteht und es somit erlaubt eine (logische) Ableitung zwischen Sätzen vorzunehmen
    }
\end{itemize}

\subsection{Prädikatenlogik erster Stufe}
\label{subsec:prädikatenlogik}

Im Gegensatz zur Aussagenlogik, welche hier nicht explizit behandelt wird, bietet die Prädikatenlogik erster Stufe eine stärkere Bindung zu der Ontologie, also der Speicherung von Informationen, welche über logische Relationen verfügen.~\cite[S. 185]{russel}

Die Prädikatenlogik erster Stufe bietet somit folgende Elemente:
\begin{itemize}
    \item{\textbf{\textit{Objekte}}

        Individuelle Entitäten
    }
    \item{\textbf{\textit{Eigenschaften}}

        welche die Entitäten voneinander unterscheiden
    }
    \item{\textbf{\textit{Relationen}}

        welche die Beziehungen zwischen den Entitäten beschreiben
    }
    \item{\textbf{\textit{Funktionen}}

        eine spezielle Form einer Relation, für welche nur eine bestimmte Ausgabe für eine bestimmte Eingabe existiert
    }
\end{itemize}

Um dies anhand eines Beispiels zu verdeutlichen:

    \noindent\hspace*{12mm}\textit{``Eine kleine Eule fliegt gerne nachts''}

Angewendet auf die Prädikatenlogik erster Stufe ergibt sich Folgendes:

\begin{itemize}
    \item{kleine: Eigenschaft}
    \item{Eule: Objekt}
    \item{fliegt: Relation}
    \item{nachts: Objekt}
\end{itemize}



% Russel \& Norvig

% Einträge im Verzeichnis erscheinen lassen ohne hier eine Referenz einzufügen
%\nocite{kopka:band1}
