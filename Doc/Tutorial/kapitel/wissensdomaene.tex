\chapter{Wissensdomäne}
\label{chap:wissensdomäne}
Wie sich in der Vorarbeit herausgestellt hat, ist es notwendig die Domäne, in welcher Anfragen gestellt werden sollen, sehr detailliert abzubilden. Zudem ist die technische Umsetzung der Suche mittels Apache Stanbol weniger weit ausgearbeitet als ursprünglich angenommen.

Um einen leicht verständlichen Einstieg in das Thema der semantischen Datenbanken / Suche bieten zu können, wird die Wissendomäne, mit welcher gearbeitet wird, stark eingeschränkt. Bei der gewählten Domäne handelt es sich um die Programmiersprache Prolog. Anhand der wird an einem exeplarischen Beispiel gezeigt, wie eine Wissendomäne modelliert wird, wie darin enthaltene Daten verknüpft werden und wie schlussendlich die logische Ableitung der Regeln der Domäne erfolgt.



