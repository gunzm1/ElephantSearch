\chapter{Inferenz und Resolution}
\label{chap:inferenz_resolution}

\section{Inferenz}
\label{sec:inferenz}
Grundsätzlich geht es bei Inferenz um den Prozess Schlussfolgerungen mithilfe von Resolution (siehe~\ref{sec:resolution}). Die logische Inferenz ist ein Prozess der Inferenz und Resolution, welcher die Folgebeziehungen zwischen Sätzen zum Ausdruck bringt.~\cite[S. 163]{russel}.

\subsection{Inferenz in Computern}
\label{subsec:inferenz-in-computer}

Das grundsätzliche Problem bei Computern im Bezug auf Inferenz ist, dass ein Computer keine Interpretation vornehmen kann und nichts über die (Um-)Welt weiss, bzw. nur, was in seiner Wissensdatenbank gespeichert ist (vgl.~\cite[S. 164]{russel}).

Angenommen, man möchte einen Computer fragen

    \noindent\hspace*{12mm}\textit{``Kann ein Vogel fliegen?''}

so weiss der Computer weder was ein Vogel ist, noch kennt er das Konzept des Fliegens. Das Einzige, was er tun kann, ist, in der Wissensdatenbank nach 

    \noindent\hspace*{12mm}\textit{``Ein Vogel kann fliegen''}

zu suchen. Findet der Computer diese Aussage in der Wissensdatenbank, so spielt es keine Rolle, dass er das Konzept des Fliegens oder Vögel nicht kennt. Die Schlussfolgerung, dass ein Vogel fliegen kann, trifft unter allen Gegebenheiten und Interpretationen zu, welche für die Wissensdatenbank zutreffen (vgl.~\cite[S.164]{russel}).

Zusammengefasst kann gesagt werden, dass die formale Inferenz in der Lage ist, gültige Schlussfolgerungen zu ziehen, auch wenn der Computer die Interpretation(en) des Anwenders nicht kennt. Der Computer zieht immer logisch gültige Schlüsse, unabhängig von der (menschlichen) Interpretation. Da der Mensch in der Regel die Interpretation kennt, erscheinen die Schlüsse dem Menschen logisch (vgl.~\cite[S. 165]{russel}).

\section{Resolution}
\label{sec:resolution}

Resolution, aus dem Lateinischen ``resolutio'', zu Detsch ``Auflösung'', ist eine Verallgemeinerung des Modus Ponens~\cite[S. 279]{russel}. Die Methode der Resolution wurde 1965 von J. A. Robinson entwickelt, dabei handelt es sich um einen vollständigen Algorithmus der Theorembeweisung für Prädikatenlogik erster Stufe.~\cite[S. 18]{russel} In der einfachsten Form handelt es sich bei der Resolution um eine Inferenz-Regel der Aussagenlogik.~\cite[S. 277]{russel}

Eine Verallgemeinerung der einfachen Form der Inferenz-Regel zur Resolution kann als Regel zur kompletten Inferenz der Prädikatenlogik erster Stufe genutzt werden.~\cite[S. 278]{russel}

\newpage

\section{Praktische Umsetzung}
\label{sec:inferenz_praktisch}

Inferenz im semantischen Netz kann grundsätzlich als das Entdecken von neuen Beziehungen zwischen Entiäten beschreiben werden. Dies heisst, dass automatische Prozeduren, in Form von so genannten \textit{Reasonern}, neue Beziehungen generieren. Wie die neuen Beziehungen umgesetzt werden --- durch Hinzufügen zu den Daten oder durch eine einfach Rückgabe dieser --- ist eine Frage der Implementation (vgl.~\cite[Abschnitt 1]{w3inference}).

\subsection{Beispiel}
\label{subsec:inferenz_beispiel}

Gegeben sei ein Datensatz, welcher die Beziehung

    \noindent\hspace*{12mm}\textit{``Prolog hatSyntaxElement Token''}

definiert. Angenommen eine Ontologie definiert die Relation

    \noindent\hspace*{12mm}\textit{``LogischesElement istUnterklasseVon SyntaxElement''}

Unterstützt nun ein Programm Inferenz, z.B. durch einen Reasoner, so kann dieser schlussfolgern, dass Prolog aus logischen Elementen besteht (vgl.~\cite[Abschnitt 'Examples']{w3inference}).

% Russel & Norvig
% Einträge im Verzeichnis erscheinen lassen ohne hier eine Referenz einzufügen
%\nocite{kopka:band1}
