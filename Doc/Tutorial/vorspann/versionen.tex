% Versionenkontrolle :
% -----------------------------------------------

\chapter*{}
\label{chap:versionen}

\begin{textblock}{180} (15,150)
\color{black}
\begin{huge}
Versionen
\end{huge}
\vspace{10mm}

\fontsize{10pt}{18pt}\selectfont
\begin{tabbing}
xxxxxxxxxxx\=xxxxxxxxxxxxxxx\=xxxxxxxxxxxxxx\=xxxxxxxxxxxxxxxxxxxxxxxxxxxxxxxxxxxxxxxxxxxxxxx \kill
\textit{Version}    \> \textit{Datum}   \> \textit{Status}      \> \textit{Bemerkungen}\\
0.1                 \> 03.10.2014       \> Entwurf              \> Initiale Erstellung des Dokumentes\\
0.2                 \> 31.10.2014       \> In Bearbeitung       \> Theoretische Grundlagen erarbeitet \\
                    \>                  \>                      \> (OWL, RDF, Expertensysteme, semantische Netze, \\
                    \>                  \>                      \> Graphen, Inferenz und Resolution)\\
0.3                 \> 01.11.2014       \> In Bearbeitung       \> Verbinden der Abschnitte (Leitfaden)\\
0.4                 \> 09.11.2014       \> In Bearbeitung       \> Kapitel SWRL und SPARQL erarbeitet\\
0.5                 \> 22.12.2014       \> In Bearbeitung       \> Beschreibung Reasoner und Tableau-Kalkül, Korrekturen\\
0.6                 \> 28.12.2014       \> In Bearbeitung       \> Korrekturen und Schlusswort\\
0.7                 \> 29.12.2014       \> Vorgelegt            \> Gegenlesen und Quellenangaben\\
1.0                 \> 30.12.2014       \> Abgestimmt           \> Fertigstellung des Dokumentes\\
\end{tabbing}

\end{textblock}
