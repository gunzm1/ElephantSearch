\chapter{Wissensdomäne}
\label{chap:wissensdomäne}
Wie sich in der Vorarbeit herausgestellt hat, ist es notwendig die Domäne, in welcher Anfragen gestellt werden sollen, sehr detailliert abzubilden. Zudem ist die technische Umsetzung der Suche mittels Apache Stanbol weniger weit ausgearbeitet als ursprünglich angenommen.

Um die Komplexität in einem angemessenen Rahmen zu halten, gilt es die Entitäten, also die Modellierung der Umwelt, stark einzuschränken.


Als Folge dieser Erkenntnisse wird die Wissendomäne, mit welcher gearbeitet wird, eingeschränkt. Bei der gewählten Domäne handelt es sich um die Grundlagen der Programmierung am Beispiel der Programmiersprache Java.

Dies erlaubt es, dass der Aufbau der Wissensdatenbank überschaubar bleibt, hat aber den Nachteil, dass nur eine beschränkte Anzahl von Fragestellungen beantwortet werden können.
