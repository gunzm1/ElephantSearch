\chapter{Ziel der Thesis}
\label{chap:thesisziel}
Als Endresultat der Thesis soll ein Dokument mit Tutorial-Charakter stehen, welches aufzeigt wie Knowledge Engineers vorgehen, um eine Problemdomäne systematisch zu modellieren und zu formalisieren.

Es soll also gezeigt werden, wie eine Speicherung von Daten, die Verknüpfung dieser sowie ihre logische Ableitungen in einer semantischen Datenbank für eine RDF/OWL-Ontologie umgesetzt wird.~\cite{Aufgabenstellung}

Dies wird mittels eines konkreten Anwendungsproblems, der Erlernung der Programmiersprache Prolog, umgesetzt.

\section{Milestones}
\label{sec:Milestones}
Die folgende Auflistung zeigt eine Übersicht, der in der Anfangsphase bereits erkennbaren Meilensteine der Arbeit:
\begin{itemize}
	\item Anforderungsdokument \\
	\item Analysieren der Modellierung der Entitäten & Attributen \\
	\item Analysieren des Erzeugen der Regeln \\
	\item Analysieren der Sprache SPARQL \\
	\item Ausformulieren der gewonnenen Erkenntnissen \\
	\item Installieren der nötigen Infrastruktur auf dem BFH Server\\
	\item Einarbeitung der Erzeugten Elemente in Stanbol \\
	\item Erzeugen einer simplen Anwenderoberfläche \\
\end{itemize}

